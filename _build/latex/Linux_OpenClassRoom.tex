%% Generated by Sphinx.
\def\sphinxdocclass{report}
\documentclass[letterpaper,10pt,french]{sphinxmanual}
\ifdefined\pdfpxdimen
   \let\sphinxpxdimen\pdfpxdimen\else\newdimen\sphinxpxdimen
\fi \sphinxpxdimen=.75bp\relax

\PassOptionsToPackage{warn}{textcomp}
\usepackage[utf8]{inputenc}
\ifdefined\DeclareUnicodeCharacter
% support both utf8 and utf8x syntaxes
\edef\sphinxdqmaybe{\ifdefined\DeclareUnicodeCharacterAsOptional\string"\fi}
  \DeclareUnicodeCharacter{\sphinxdqmaybe00A0}{\nobreakspace}
  \DeclareUnicodeCharacter{\sphinxdqmaybe2500}{\sphinxunichar{2500}}
  \DeclareUnicodeCharacter{\sphinxdqmaybe2502}{\sphinxunichar{2502}}
  \DeclareUnicodeCharacter{\sphinxdqmaybe2514}{\sphinxunichar{2514}}
  \DeclareUnicodeCharacter{\sphinxdqmaybe251C}{\sphinxunichar{251C}}
  \DeclareUnicodeCharacter{\sphinxdqmaybe2572}{\textbackslash}
\fi
\usepackage{cmap}
\usepackage[T1]{fontenc}
\usepackage{amsmath,amssymb,amstext}
\usepackage{babel}
\usepackage{times}
\usepackage[Sonny]{fncychap}
\ChNameVar{\Large\normalfont\sffamily}
\ChTitleVar{\Large\normalfont\sffamily}
\usepackage{sphinx}

\fvset{fontsize=\small}
\usepackage{geometry}

% Include hyperref last.
\usepackage{hyperref}
% Fix anchor placement for figures with captions.
\usepackage{hypcap}% it must be loaded after hyperref.
% Set up styles of URL: it should be placed after hyperref.
\urlstyle{same}

\addto\captionsfrench{\renewcommand{\figurename}{Fig.}}
\addto\captionsfrench{\renewcommand{\tablename}{Tableau}}
\addto\captionsfrench{\renewcommand{\literalblockname}{Code source}}

\addto\captionsfrench{\renewcommand{\literalblockcontinuedname}{suite de la page précédente}}
\addto\captionsfrench{\renewcommand{\literalblockcontinuesname}{suite sur la page suivante}}
\addto\captionsfrench{\renewcommand{\sphinxnonalphabeticalgroupname}{Non-alphabetical}}
\addto\captionsfrench{\renewcommand{\sphinxsymbolsname}{Symboles}}
\addto\captionsfrench{\renewcommand{\sphinxnumbersname}{Numbers}}

\addto\extrasfrench{\def\pageautorefname{page}}





\title{Linux\_OpenClassRoom Documentation}
\date{nov. 04, 2018}
\release{}
\author{Eric}
\newcommand{\sphinxlogo}{\vbox{}}
\renewcommand{\releasename}{}
\makeindex
\begin{document}

\ifdefined\shorthandoff
  \ifnum\catcode`\=\string=\active\shorthandoff{=}\fi
  \ifnum\catcode`\"=\active\shorthandoff{"}\fi
\fi

\pagestyle{empty}
\maketitle
\pagestyle{plain}
\sphinxtableofcontents
\pagestyle{normal}
\phantomsection\label{\detokenize{index::doc}}



\chapter{Raccourcis clavier divers}
\label{\detokenize{01-shortcuts:raccourcis-clavier-divers}}\label{\detokenize{01-shortcuts::doc}}\begin{description}
\item[{\sphinxcode{\sphinxupquote{Ctrl + R}}}] \leavevmode
search for previous command

\item[{\sphinxcode{\sphinxupquote{Ctrl + L}}}] \leavevmode
efface le contenu de la console. Utile pour faire un peu de ménage quand votre console est encombrée, ou quand votre boss passe derrière vous et que vous n’aimeriez pas qu’il voie ce que vous étiez en train de faire. À noter qu’il existe aussi une commande, clear, qui fait exactement la même chose.

\item[{\sphinxcode{\sphinxupquote{Ctrl + D}}}] \leavevmode
envoie le message EOF (fin de fichier) à la console. Si vous tapez ce raccourci dans une ligne de commande vide (c’est-à-dire sans avoir écrit un début de commande au préalable), cela fermera la console en cours. À noter qu’il existe aussi la commande exit qui a le même effet.

\item[{\sphinxcode{\sphinxupquote{Shift + PgUp}}}] \leavevmode
vous permet de « remonter » dans les messages envoyés par la console. En mode graphique, la molette de la souris accomplit aussi très bien cette action. La touche Page Up est généralement représentée sur votre clavier par une flèche directionnelle Haut barrée de plusieurs petites lignes horizontales.

\item[{\sphinxcode{\sphinxupquote{Shift + PgDown}}}] \leavevmode
pareil, mais pour redescendre.

\item[{\sphinxcode{\sphinxupquote{Ctrl + A}}}] \leavevmode
ramène le curseur au début de la commande. La touche Origine a le même effet (elle est située à côté de la touche Fin et représentée par une flèche pointant en haut à gauche).

\item[{\sphinxcode{\sphinxupquote{Ctrl + E}}}] \leavevmode
ramène le curseur à la fin de la ligne de commandes. La touche Fin a le même effet.

\item[{\sphinxcode{\sphinxupquote{Ctrl + U}}}] \leavevmode
supprime tout ce qui se trouve à gauche du curseur. Si celui-ci est situé à la fin de la ligne, cette dernière sera donc supprimée.

\item[{\sphinxcode{\sphinxupquote{Ctrl + K}}}] \leavevmode
supprime tout ce qui se trouve à droite du curseur. S’il est situé au début de la ligne, celle-ci sera donc totalement supprimée.

\item[{\sphinxcode{\sphinxupquote{Ctrl + W}}}] \leavevmode
supprime le premier mot situé à gauche du curseur. Un « mot » est séparé par des espaces ; on s’en sert en général pour supprimer le paramètre situé à gauche du curseur.

\item[{\sphinxcode{\sphinxupquote{Ctrl + Y}}}] \leavevmode
si vous avez supprimé du texte avec une des commandes Ctrl + U, Ctrl + K ou Ctrl + W qu’on vient de voir, alors le raccourci Ctrl + Y « collera » le texte que vous venez de supprimer. C’est donc un peu comme un couper-coller.

\end{description}


\chapter{Navigation dans les dossiers}
\label{\detokenize{02-navigation:navigation-dans-les-dossiers}}\label{\detokenize{02-navigation::doc}}\begin{description}
\item[{\sphinxcode{\sphinxupquote{pwd}}}] \leavevmode
savoir où je suis

\item[{\sphinxcode{\sphinxupquote{which cmd}}}] \leavevmode
permet de localiser l’emplacement d’une commande cmd

\item[{\sphinxcode{\sphinxupquote{ls -F}}}] \leavevmode
rajoute à la fin des éléments un symbole pour qu’on puise faire la distinction entre les dossiers, fichiers, raccourcis…

\item[{\sphinxcode{\sphinxupquote{ls -lh}}}] \leavevmode
afficher la taille des éléments (-h : « for humans »)

\item[{\sphinxcode{\sphinxupquote{cd}}}] \leavevmode
sans arguments, ramène dans home/currentUser/

\item[{\sphinxcode{\sphinxupquote{du}}}] \leavevmode
pour Disk Usage (utilisation du disque) vous donne des informations sur la taille qu’occupent les dossiers sur votre disque

\item[{\sphinxcode{\sphinxupquote{du -c}}}] \leavevmode
produces a grand total

\item[{\sphinxcode{\sphinxupquote{du -hsc *}}}] \leavevmode
This finds the size recursively and puts it next to each folder name, along with total size at the bottom, all in the human format.

\item[{\sphinxcode{\sphinxupquote{du -hs * \textbar{} sort -h}}}] \leavevmode
same as previous but sort by size

\end{description}


\chapter{Manipuler les fichiers}
\label{\detokenize{03-manipuler:manipuler-les-fichiers}}\label{\detokenize{03-manipuler::doc}}\begin{description}
\item[{\sphinxcode{\sphinxupquote{tree -{-}du -h}}}] \leavevmode
returns a tree-like representation of the current folder and its subfilders/files (could be a huge dump)

Note that tree might have to be installed first with \sphinxcode{\sphinxupquote{sudo apt-get install tree}}

\item[{\sphinxcode{\sphinxupquote{less -N fichier.txt}}}] \leavevmode
affiche le contenu d’un fichier avec numéro de lignes

\item[{\sphinxcode{\sphinxupquote{touch fichier.txt}}}] \leavevmode
crée un fichier vide

\item[{\sphinxcode{\sphinxupquote{head/tail}}}] \leavevmode
affiche le début/fin d’un fichier

\item[{\sphinxcode{\sphinxupquote{tail -f fichier}}}] \leavevmode
permet de voir l’évolution de la fin d’un fichier (par ex : var/log/syslog)

\item[{\sphinxcode{\sphinxupquote{.}}}] \leavevmode
le dossier où je me trouve

\item[{\sphinxcode{\sphinxupquote{cp -R dossierSource dossierDest}}}] \leavevmode
Avec l’option \sphinxcode{\sphinxupquote{-R}} (un « R » majuscule !), vous pouvez copier un dossier, ainsi que tous les sous-dossiers et fichiers qu’il contient !

\item[{\sphinxcode{\sphinxupquote{mv}}}] \leavevmode
déplacer

\item[{\sphinxcode{\sphinxupquote{rm fichier}}}] \leavevmode
suprimer

\item[{\sphinxcode{\sphinxupquote{rm -i fichier}}}] \leavevmode
supprimer avec confirmation

\item[{\sphinxcode{\sphinxupquote{rm -f fichier}}}] \leavevmode
forcer la suppression quoi qu’il arrive

\item[{\sphinxcode{\sphinxupquote{rm -v fichier}}}] \leavevmode
supprimer avec verbose

\item[{\sphinxcode{\sphinxupquote{rm - r dossier/}}}] \leavevmode
supprimer un dossier et son contenu

\item[{\sphinxcode{\sphinxupquote{rm -rf *}}}] \leavevmode
supprimer tous les fichiers du dossier courant

ATTENTION : NE SURTOUT PAS CONFONDRE AVEC \sphinxcode{\sphinxupquote{rm -rf /*}} qui supprime tout, partout et flingue tout le système

\item[{\sphinxcode{\sphinxupquote{ln fichier1 fichier2}}}] \leavevmode
crée un lien physique entre fichier1 et fichier2 (même « inode », pointent vers le même contenu)

\item[{\sphinxcode{\sphinxupquote{ln -s fichier1 fichier2}}}] \leavevmode
crée un lien symbolique entre fichier1 et fichier2 (\textasciitilde{}raccourci Windows)

\item[{\sphinxcode{\sphinxupquote{mkdir nom\_du\_dossier}}}] \leavevmode
créer un dossier

\item[{\sphinxcode{\sphinxupquote{mkdir -p dossier/\{dossier1,dossier2,dossier3\}}}}] \leavevmode
crée l’arborescence suivante :

\fvset{hllines={, ,}}%
\begin{sphinxVerbatim}[commandchars=\\\{\}]
├── \PYG{o}{[}eric     \PYG{l+m}{4}.0K\PYG{o}{]}  dossier
│   ├── \PYG{o}{[}eric     \PYG{l+m}{4}.0K\PYG{o}{]}  dossier1
│   ├── \PYG{o}{[}eric     \PYG{l+m}{4}.0K\PYG{o}{]}  dossier2
│   └── \PYG{o}{[}eric     \PYG{l+m}{4}.0K\PYG{o}{]}  dossier3
\end{sphinxVerbatim}

\sphinxcode{\sphinxupquote{-p}}, \sphinxcode{\sphinxupquote{-{-}parents}} : no error if existing, make parent directories as needed

\end{description}


\chapter{Les flux de redirection}
\label{\detokenize{04-redirection:les-flux-de-redirection}}\label{\detokenize{04-redirection::doc}}\begin{description}
\item[{\sphinxcode{\sphinxupquote{cmd \textgreater{} fichier.txt}}}] \leavevmode
redirige la sortie standard de la commande cmd vers le fichier fichier.txt

\item[{\sphinxcode{\sphinxupquote{cmd \textgreater{}\textgreater{} fichier.txt}}}] \leavevmode
redirige la sortie standard de la commande cmd vers la fin du fichier fichier.txt

\item[{\sphinxcode{\sphinxupquote{cmd \textgreater{} fichier.txt 2\textgreater{} erreurs.txt}}}] \leavevmode
redirige la sortie standard de la commande cmd vers le fichier fichier.txt et la sortie des erreurs vers le fichier erreurs.txt (fin du fichier : 2\textgreater{}\textgreater{})

\item[{\sphinxcode{\sphinxupquote{cmd \textgreater{} fichier.txt 2\textgreater{}\$1}}}] \leavevmode
redirige la sortie standard ET celle des erreurs de la commande cmd vers le fichier fichier.txt

\item[{\sphinxcode{\sphinxupquote{cmd \textless{} fichier.txt}}}] \leavevmode
envoie le contenu de fichier.txt à la commande cmd

\item[{\sphinxcode{\sphinxupquote{cmd \textless{}\textless{} motDeFin}}}] \leavevmode
passe la console en mode saisie au clavier, ligne par ligne. Toutes ces lignes seront envoyées à la commande lorsque le mot-clé de fin « motDeFin » aura été écrit.

\item[{\sphinxcode{\sphinxupquote{cmd1 \textbar{} cmd2}}}] \leavevmode
exécute cmd1 et envoie sa sortie en entrée de cmd2 (« pipe »)

\item[{\sphinxcode{\sphinxupquote{cmd \textgreater{} /dev/null 2\textgreater{}\&1}}}] \leavevmode
démarre le programme sans récupérer sa sortie ni ses erreurs, qui sont « jetées » dans /dev/null

\end{description}


\chapter{Surveiller l’activité du système}
\label{\detokenize{05-activite:surveiller-l-activite-du-systeme}}\label{\detokenize{05-activite::doc}}\begin{description}
\item[{\sphinxcode{\sphinxupquote{w}}}] \leavevmode
permet de voir qui fait quoi
\begin{quote}

\sphinxcode{\sphinxupquote{USER}} : le nom de l’utilisateur (son login) ;

\sphinxcode{\sphinxupquote{TTY}} : le nom de la console dans laquelle se trouve l’utilisateur. Souvenez-vous que sous Linux il y a en général six consoles (tty1 à tty6) et qu’en plus de ça, on peut en ouvrir une infinité grâce aux consoles graphiques (leur nom commence par pts, en général), comme le propose le programme « Terminal » sous Unity ou « Konsole » sous KDE ;

\sphinxcode{\sphinxupquote{FROM}} : c’est l’adresse IP (ou le nom d’hôte) depuis laquelle il se connecte. Ici, comme je me suis connecté en local (sur ma propre machine, sans passer par Internet), il n’y a pas vraiment d’IP ;

\sphinxcode{\sphinxupquote{LOGIN@}} : l’heure à laquelle cet utilisateur s’est connecté ;

\sphinxcode{\sphinxupquote{IDLE}} : depuis combien de temps cet utilisateur est inactif (depuis combien de temps il n’a pas lancé de commande)

\sphinxcode{\sphinxupquote{WHAT}} : la commande qu’il est en train d’exécuter en ce moment. En général, si vous voyez bash, cela signifie que l’invite de commandes est ouverte et qu’aucune commande particulière n’est exécutée.
\end{quote}

\end{description}

\sphinxcode{\sphinxupquote{ps}} : liste statique des processus
\begin{description}
\item[{\sphinxcode{\sphinxupquote{ps -ef}}}] \leavevmode
lister tous les processus (\textasciitilde{}ps -A)

\item[{\sphinxcode{\sphinxupquote{ps -ejH}}}] \leavevmode
afficher les processus en arbre

\item[{\sphinxcode{\sphinxupquote{ps -u UTILISATEUR}}}] \leavevmode
lister les processus lancés par un utilisateur

\item[{\sphinxcode{\sphinxupquote{top}}}] \leavevmode
liste dynamique des processus
\begin{quote}

\sphinxcode{\sphinxupquote{q}} : ferme top ;

\sphinxcode{\sphinxupquote{h}} : affiche l’aide, et donc la liste des touches utilisables.

\sphinxcode{\sphinxupquote{B}} : met en gras certains éléments.

\sphinxcode{\sphinxupquote{f}} : ajoute ou supprime des colonnes dans la liste.

\sphinxcode{\sphinxupquote{F}} : change la colonne selon laquelle les processus sont triés. En général, laisser le tri par défaut en fonction de \%CPU est suffisant.

\sphinxcode{\sphinxupquote{u}} : filtre en fonction de l’utilisateur que vous voulez.

\sphinxcode{\sphinxupquote{k}} : tue un processus, c’est-à-dire arrête ce processus. Ne vous inquiétez pas, en général les processus ne souffrent pas. On vous demandera le numéro (PID) du processus que vous voulez tuer. Nous reviendrons sur l’arrêt des processus un peu plus loin.

\sphinxcode{\sphinxupquote{s}} : change l’intervalle de temps entre chaque rafraîchissement de la liste (par défaut, c’est toutes les trois secondes).
\end{quote}

\item[{\sphinxcode{\sphinxupquote{kill processPid}}}] \leavevmode
tue le processus dont le PID est processPid (proprement)

\item[{\sphinxcode{\sphinxupquote{kill -9 processPid}}}] \leavevmode
force l’arrêt immédiat du processus processPid (bourrin)

\item[{\sphinxcode{\sphinxupquote{killall processName}}}] \leavevmode
tue plusieurs processus dont le com est processName

\item[{\sphinxcode{\sphinxupquote{sudo halt/reboot}}}] \leavevmode
arrête/reboot le PC

\end{description}


\chapter{Exécuter des programmes en arrière-plan}
\label{\detokenize{06-arriere_plan:executer-des-programmes-en-arriere-plan}}\label{\detokenize{06-arriere_plan::doc}}\begin{description}
\item[{\sphinxcode{\sphinxupquote{cmd \&}}}] \leavevmode
démarre la commande cmd en arrière plan

\item[{\sphinxcode{\sphinxupquote{cmd \textgreater{} /dev/null 2\textgreater{}\&1 \&}}}] \leavevmode
démarre le programme en arrière plan sans récupérer sa sortie ni ses erreurs, qui sont « jetées » dans /dev/null

\item[{\sphinxcode{\sphinxupquote{nohup cmd}}}] \leavevmode
détacher le processus de la console (ave les autres méthodes le programme se ferme quand on ferme la console, là pas)

\item[{\sphinxcode{\sphinxupquote{Ctrl + z}}}] \leavevmode
mettre en pause l’exécution du programme

\item[{\sphinxcode{\sphinxupquote{bg}}}] \leavevmode
fait passer en arrière plan le processus que l’on a stoppé avec Ctrl+z

\item[{\sphinxcode{\sphinxupquote{jobs}}}] \leavevmode
affiche les processus en arrière plan

\item[{\sphinxcode{\sphinxupquote{fg}}}] \leavevmode
reprendre un processus au premier plan (foreground)

\item[{\sphinxcode{\sphinxupquote{fg \%2}}}] \leavevmode
reprendre le processus n°2 (trouvé avec la commande jobs) en premier plan

\item[{\sphinxcode{\sphinxupquote{screen}}}] \leavevmode
permet d’avoir plusieurs consoles en une (installer : sudo apt-get install screen)

\sphinxcode{\sphinxupquote{Ctrl + a puis c}} : créer une nouvelle « fenêtre ».

\sphinxcode{\sphinxupquote{Ctrl + a puis w}} : afficher la liste des « fenêtres » actuellement ouvertes. En bas de l’écran vous verrez par exemple apparaître : 0-\$ bash  1*\$ bash. Cela signifie que vous avez deux fenêtres ouvertes, l’une numérotée 0, l’autre 1. Celle sur laquelle vous vous trouvez actuellement contient une étoile * (on se trouve donc ici dans la fenêtre n° 1).

\sphinxcode{\sphinxupquote{Ctrl + a puis A}} : renommer la fenêtre actuelle. Ce nom apparaît lorsque vous affichez la liste des fenêtres avec Ctrl + a puis w.

\sphinxcode{\sphinxupquote{Ctrl + a puis n}} : passer à la fenêtre suivante (next).

\sphinxcode{\sphinxupquote{Ctrl + a puis p}} : passer à la fenêtre précédente (previous).

\sphinxcode{\sphinxupquote{Ctrl + a puis Ctrl + a}} : revenir à la dernière fenêtre utilisée.

\sphinxcode{\sphinxupquote{Ctrl + a puis un chiffre de 0 à 9}} : passer à la fenêtre n° X.

\sphinxcode{\sphinxupquote{Ctrl + a puis "}} : choisir la fenêtre dans laquelle on veut aller.

\sphinxcode{\sphinxupquote{Ctrl + a puis k}} : fermer la fenêtre actuelle (kill).
\begin{quote}

\sphinxcode{\sphinxupquote{Ctrl + a puis S}} : découper screen en plusieurs parties (split)

\sphinxcode{\sphinxupquote{Ctrl + a puis d}} : détache screen et vous permet de retrouver l’invite de commandes « normale » sans arrêter screen. C’est peut-être une des fonctionnalités les plus utiles que nous devons approfondir, et cela nous ramène d’ailleurs à l’exécution de programmes en arrière-plan dont nous avons parlé au début du chapitre.
\end{quote}

\item[{\sphinxcode{\sphinxupquote{screen -r}}}] \leavevmode
récupérer son ancienne session screen (détachée)

\item[{\sphinxcode{\sphinxupquote{screen -ls}}}] \leavevmode
affiche la liste des screens actuellement ouverts

\end{description}


\chapter{Exécuter un programme à une heure différée}
\label{\detokenize{07-differe:executer-un-programme-a-une-heure-differee}}\label{\detokenize{07-differe::doc}}\begin{description}
\item[{\sphinxcode{\sphinxupquote{date}}}] \leavevmode
afficher / régler l’heure

\item[{\sphinxcode{\sphinxupquote{at HH:MM}}}] \leavevmode
donne la possibilité de démarrer un programme à HH:MM. Un promppt aparait pour demander lequel.

\end{description}

\sphinxcode{\sphinxupquote{at HH:MM tomorrow}}
\begin{description}
\item[{\sphinxcode{\sphinxupquote{at HH:MM 11/15/10}}}] \leavevmode
attention date au format américain

\end{description}

\sphinxcode{\sphinxupquote{at now +5 minutes}}
\begin{description}
\item[{\sphinxcode{\sphinxupquote{atq}}}] \leavevmode
afficher les jobs en attente -\textgreater{} donne un n°

\item[{\sphinxcode{\sphinxupquote{atrm x}}}] \leavevmode
supprimer un job en attente dont le numéro est x

\item[{\sphinxcode{\sphinxupquote{sleep x}}}] \leavevmode
attend x secondes (minute : xm, heure : xh, jour : xd)

\item[{\sphinxcode{\sphinxupquote{echo "export EDITOR=nano" \textgreater{}\textgreater{}  \textasciitilde{}/.bashrc}}}] \leavevmode
faire de Nano l’éditeur par défaut

\item[{\sphinxcode{\sphinxupquote{crontab -e}}}] \leavevmode
modifier la crontab

\item[{\sphinxcode{\sphinxupquote{crontab -l}}}] \leavevmode
afficher la crontab

\item[{\sphinxcode{\sphinxupquote{crontab -r}}}] \leavevmode
supprimer la crontab (immédiate et sans confirmation)

\item[{\sphinxcode{\sphinxupquote{m     h       dom     mon     dow     command}}}] \leavevmode\begin{itemize}
\item {} 
\sphinxcode{\sphinxupquote{m}} : minutes (0 - 59)

\item {} 
\sphinxcode{\sphinxupquote{h}} : heures (0 - 23)

\item {} 
\sphinxcode{\sphinxupquote{dom}} (day of month) : jour du mois (1 - 31)

\item {} 
\sphinxcode{\sphinxupquote{mon}} (month) : mois (1 - 12) ;

\item {} 
\sphinxcode{\sphinxupquote{dow}} (day of week) : jour de la semaine (0 - 6, 0 étant le dimanche) ;

\item {} 
\sphinxcode{\sphinxupquote{command}} : c’est la commande à exécuter.

\end{itemize}

ex : \sphinxcode{\sphinxupquote{47 15 * * * touch /home/mateo21/fichier.txt}} -\textgreater{} tous les jours à 15h47
\begin{description}
\item[{Pour chaque champ, on a le droit à différentes notations :}] \leavevmode\begin{itemize}
\item {} 
\sphinxcode{\sphinxupquote{5}} (un nombre) : exécuté lorsque le champ prend la valeur 5

\item {} 
\sphinxcode{\sphinxupquote{*}} : exécuté tout le temps (toutes les valeurs sont bonnes) ;

\item {} 
\sphinxcode{\sphinxupquote{3,5,10}} : exécuté lorsque le champ prend la valeur 3, 5 ou 10. Ne pas mettre d’espace après la virgule ;

\item {} 
\sphinxcode{\sphinxupquote{3-7}} : exécuté pour les valeurs 3 à 7 ;

\item {} 
\sphinxcode{\sphinxupquote{*/3}} : exécuté tous les multiples de 3 (par exemple à 0 h, 3 h, 6 h, 9 h…).

\end{itemize}

\end{description}

\end{description}


\chapter{Extraire, trier et filtrer des données}
\label{\detokenize{08-trier:extraire-trier-et-filtrer-des-donnees}}\label{\detokenize{08-trier::doc}}\begin{description}
\item[{\sphinxcode{\sphinxupquote{grep -i mot fichier}}}] \leavevmode
affiches les occurences de « mot » dans fichier sans faire attention à la casse.

\item[{\sphinxcode{\sphinxupquote{grep -n mot fichier}}}] \leavevmode
affiche les n° de ligne

\item[{\sphinxcode{\sphinxupquote{grep -v mot fichier}}}] \leavevmode
inversion de la recherche : « tout ce qui ne contient PAS mot »

\item[{\sphinxcode{\sphinxupquote{grep -r mot répertoire}}}] \leavevmode
rechercher dans tous les fichiers et sous-dossiers (équivalent à rgrep)

\end{description}

\sphinxcode{\sphinxupquote{grep "ma phrase contient des espaces" monFichier}}
\begin{description}
\item[{\sphinxcode{\sphinxupquote{grep -E mot fichier}}}] \leavevmode
grep avec expression régulière
\begin{description}
\item[{Expressions régulières :}] \leavevmode\begin{itemize}
\item {} 
\sphinxcode{\sphinxupquote{.}} : Caractère quelconque

\item {} 
\sphinxcode{\sphinxupquote{\textasciicircum{}}} : Début de ligne (cherche un mot placé en début de ligne)

\item {} 
\sphinxcode{\sphinxupquote{\$}} : Fin de ligne (cherche un mot placé en fin de ligne)

\item {} 
\sphinxcode{\sphinxupquote{{[}{]}}} : Un des caractères entre les crochets

\item {} 
\sphinxcode{\sphinxupquote{?}} : L’élément précédent est optionnel (peut être présent 0 ou 1 fois)

\item {} 
\sphinxcode{\sphinxupquote{*}} : L’élément précédent peut être présent 0, 1 ou plusieurs fois

\item {} 
\sphinxcode{\sphinxupquote{+}} : L’élément précédent doit être présent 1 ou plusieurs fois

\item {} 
\sphinxcode{\sphinxupquote{\textbar{}}} : Ou

\item {} 
\sphinxcode{\sphinxupquote{()}} : Groupement d’expressions

\end{itemize}

\end{description}

\item[{\sphinxcode{\sphinxupquote{sort fichier}}}] \leavevmode
trier le contenu d’un fichier

\item[{\sphinxcode{\sphinxupquote{sort -o noms\_tries.txt noms.txt}}}] \leavevmode
avec sortie vers noms\_tries.txt

\item[{\sphinxcode{\sphinxupquote{sort -R fichier}}}] \leavevmode
trier aléatoirement

\item[{\sphinxcode{\sphinxupquote{sort -n fichier}}}] \leavevmode
trier des nombres (ne se bas pas sur l’alphabet, sinon : 1 123 23 …)

\item[{\sphinxcode{\sphinxupquote{wc fichier.txt}}}] \leavevmode\begin{description}
\item[{renvoie un résultat type « a b c fichier.txt » où :}] \leavevmode\begin{itemize}
\item {} 
\sphinxcode{\sphinxupquote{a}} : nb de lignes (\sphinxcode{\sphinxupquote{-l}})

\item {} 
\sphinxcode{\sphinxupquote{b}} : nb de mots (\sphinxcode{\sphinxupquote{-w}})

\item {} 
\sphinxcode{\sphinxupquote{c}} : nb d’octets (\sphinxcode{\sphinxupquote{-c}})

\end{itemize}

\end{description}

\item[{\sphinxcode{\sphinxupquote{wc -m fichier.txt}}}] \leavevmode
nb de caractères dans le fichier

\item[{\sphinxcode{\sphinxupquote{uniq fichier.txt}}}] \leavevmode
supprime les doublons

\item[{\sphinxcode{\sphinxupquote{uniq doublons.txt sans\_doublons.txt}}}] \leavevmode
sort ça dans sans\_doublons.txt

\item[{\sphinxcode{\sphinxupquote{uniq -c}}}] \leavevmode
compte le nb d’occurences
\begin{description}
\item[{ex :}] \leavevmode
\sphinxcode{\sphinxupquote{uniq -c doublons.txt}}

\end{description}

résultats :

\fvset{hllines={, ,}}%
\begin{sphinxVerbatim}[commandchars=\\\{\}]
\PYG{l+m}{1} Albert
\PYG{l+m}{3} François
\PYG{l+m}{1} Jean
\PYG{l+m}{2} Marcel
\end{sphinxVerbatim}

\item[{\sphinxcode{\sphinxupquote{uniq -d fichier}}}] \leavevmode
uniquement les lignes en double

\item[{\sphinxcode{\sphinxupquote{cut -c 2-5 noms.txt}}}] \leavevmode
conserve uniquement les caractères 2 à 5 de chaque ligne

\item[{\sphinxcode{\sphinxupquote{cut -c -3 noms.txt}}}] \leavevmode
conserve uniquement les caractères 1 à 3 de chaque ligne

\item[{\sphinxcode{\sphinxupquote{cut -c 3- noms.txt}}}] \leavevmode
du n°3 au dernier de chaque ligne

\item[{\sphinxcode{\sphinxupquote{cut -d , -f 1 notes.csv}}}] \leavevmode\begin{itemize}
\item {} 
\sphinxcode{\sphinxupquote{-d}} : indique quel est le délimiteur dans le fichier (ici “,”)

\item {} 
\sphinxcode{\sphinxupquote{-f}} : indique le numéro du ou des champs à couper, cad que l’on garde (ici le 1er)

\end{itemize}

\item[{\sphinxcode{\sphinxupquote{cut -d , -f 1,3 notes.csv}}}] \leavevmode
garde les champs 1 ET 3

\item[{\sphinxcode{\sphinxupquote{cut -d , -f 1-3 notes.csv}}}] \leavevmode
garde les champs 1 à 3

\end{description}


\chapter{Les utilisateurs et les droits}
\label{\detokenize{09-utilisateurs:les-utilisateurs-et-les-droits}}\label{\detokenize{09-utilisateurs::doc}}\begin{description}
\item[{\sphinxcode{\sphinxupquote{adduser userName}}}] \leavevmode
ajouter un utilisateur userName

\item[{\sphinxcode{\sphinxupquote{deluser userName}}}] \leavevmode
supprimer un utilisateur userName

\item[{\sphinxcode{\sphinxupquote{deluser -{-}remove-home userName}}}] \leavevmode
supprime aussi son répertoire personnel

\item[{\sphinxcode{\sphinxupquote{pswd userName}}}] \leavevmode
changer le mot de passe de userName

\item[{\sphinxcode{\sphinxupquote{addgroup groupName}}}] \leavevmode
crée un groupe d’utilisateurs

\item[{\sphinxcode{\sphinxupquote{usermod -l userName}}}] \leavevmode
modifier le nom de l’utilisateur

\item[{\sphinxcode{\sphinxupquote{usermod -g groupName userName}}}] \leavevmode
modifier le groupe d’un utilisateur (remplace les précédents groupes)

\item[{\sphinxcode{\sphinxupquote{usermod -G amis,paris,collegues patrick}}}] \leavevmode
ajouter l’utilisateur à plusieurs groupes (G majuscule) (remplace les précédents groupes)

\item[{\sphinxcode{\sphinxupquote{usermod -aG amis,paris,collegues patrick}}}] \leavevmode
ajoute l’utilisateur aux groupes en gardant ses groupes précédents

\item[{\sphinxcode{\sphinxupquote{delgroup groupName}}}] \leavevmode
supprime un groupe

\item[{\sphinxcode{\sphinxupquote{chown userName fichier}}}] \leavevmode
rend patrick propriétaire de rapport.txt

\item[{\sphinxcode{\sphinxupquote{chgrp groupName fichier}}}] \leavevmode
rend le groupe groupName propriétaire du fichier

\item[{\sphinxcode{\sphinxupquote{chown userName:groupName fichier}}}] \leavevmode
Cela affectera le fichier à l’utilisateur userName et au groupe groupName.

\item[{\sphinxcode{\sphinxupquote{chown -R userName:userName /dossier/}}}] \leavevmode
modifie tous les sous-dossiers et fichiers contenus dans un dossier pour y affecter un nouvel utilisateur (et un nouveau groupe si on utilise la technique du deux points)

\item[{ex}] \leavevmode{[}\sphinxcode{\sphinxupquote{drwxr-xr-x 2 mateo21 mateo21 4096 2007-11-13 21:53 Desktop}}{]}\begin{itemize}
\item {} 
\sphinxcode{\sphinxupquote{d}} (Directory) : indique si l’élément est un dossier ;

\item {} 
\sphinxcode{\sphinxupquote{l}} (Link) : indique si l’élément est un lien (raccourci) ;

\item {} 
\sphinxcode{\sphinxupquote{r}} (Read) : indique si on peut lire l’élément ;

\item {} 
\sphinxcode{\sphinxupquote{w}} (Write) : indique si on peut modifier l’élément ;

\item {} 
\sphinxcode{\sphinxupquote{x}} (eXecute) : si c’est un fichier, « x » indique qu’on peut l’exécuter. Ce n’est utile que pour les fichiers exécutables (programmes et scripts).

\end{itemize}

Si c’est un dossier, « x » indique qu’on peut le « traverser », c’est-à-dire qu’on peut voir les sous-dossiers qu’il contient si on a le droit de lecture dessus.
\begin{itemize}
\item {} 
le premier triplet rwx indique les droits que possède le propriétaire du fichier sur ce dernier ;

\item {} 
le second triplet rwx indique les droits que possèdent les autres membres du groupe sur ce fichier ;

\item {} 
enfin, le dernier triplet rwx indique les droits que possèdent tous les autres utilisateurs de la machine sur le fichier.

\end{itemize}

\item[{\sphinxcode{\sphinxupquote{chmod}}}] \leavevmode\begin{description}
\item[{modifie les droits -\textgreater{} utiliser des numéros :}] \leavevmode\begin{itemize}
\item {} 
\sphinxcode{\sphinxupquote{r}} = 4

\item {} 
\sphinxcode{\sphinxupquote{w}} = 2

\item {} 
\sphinxcode{\sphinxupquote{x}} = 1

\end{itemize}

donc :

\end{description}
\begin{itemize}
\item {} 
\sphinxcode{\sphinxupquote{-{-}-}} = 0+0+0 = 0

\item {} 
\sphinxcode{\sphinxupquote{r-{-}}} = 4+0+0 = 4

\item {} 
\sphinxcode{\sphinxupquote{-w-}} = 0+2+0 = 2

\item {} 
\sphinxcode{\sphinxupquote{-{-}x}} = 0+0+1 = 1

\item {} 
\sphinxcode{\sphinxupquote{rw-}} = 4+2+0 = 6

\item {} 
\sphinxcode{\sphinxupquote{-wx}} = 0+2+1 = 3

\item {} 
\sphinxcode{\sphinxupquote{r-x}} = 4+0+1 = 5

\item {} 
\sphinxcode{\sphinxupquote{rwx}} = 4+2+1 = 7

\end{itemize}

\item[{\sphinxcode{\sphinxupquote{chmod 600 rapport.txt}}}] \leavevmode
\sphinxcode{\sphinxupquote{-rw-{-}-{-}-{-}- 1 mateo21 mateo21 0 2007-11-15 23:14 rapport.txt}}

\end{description}


\chapter{Nano : éditeur de texte}
\label{\detokenize{10-nano:nano-editeur-de-texte}}\label{\detokenize{10-nano::doc}}\begin{description}
\item[{\sphinxcode{\sphinxupquote{nano -m}}}] \leavevmode
autorise la souris

\item[{\sphinxcode{\sphinxupquote{nano -i}}}] \leavevmode
indentation autorise

\item[{\sphinxcode{\sphinxupquote{nano -A}}}] \leavevmode
active le retour intelligent au début de la ligne. Normalement, lorsque vous appuyez sur la touche Origine(aussi connue sous le nom de Home) située à côté de la touche Fin, le curseur se repositionne au tout début de la ligne. Avec cette commande, il se positionnera après les alinéas.

\item[{\sphinxcode{\sphinxupquote{Alt + Shift + 6}}}] \leavevmode
Copier une ligne

\end{description}

Options par défaut : ouvrir \sphinxcode{\sphinxupquote{.nanosrc}}

\fvset{hllines={, ,}}%
\begin{sphinxVerbatim}[commandchars=\\\{\}]
\PYG{n+nb}{set} mouse
    \PYG{n+nb}{set} autoindent
    \PYG{n+nb}{set} smarthome
\end{sphinxVerbatim}


\chapter{Alias}
\label{\detokenize{11-alias:alias}}\label{\detokenize{11-alias::doc}}\begin{description}
\item[{créer un alias dans \sphinxcode{\sphinxupquote{.bashrc}} :}] \leavevmode
ex : \sphinxcode{\sphinxupquote{alias ll='ls -l'}}

\end{description}


\chapter{Installer des programmes avec apt-get}
\label{\detokenize{12-apt-get:installer-des-programmes-avec-apt-get}}\label{\detokenize{12-apt-get::doc}}
\sphinxcode{\sphinxupquote{/etc/apt/sources.list}} :
\begin{itemize}
\item {} 
adresses des serveurs pour les dépôts

\item {} 
pour autoriser le téléchargement de sources

\end{itemize}

ex :

\fvset{hllines={, ,}}%
\begin{sphinxVerbatim}[commandchars=\\\{\}]
deb\PYGZhy{}src http://raspbian.raspberrypi.org/raspbian/ stretch main contrib non\PYGZhy{}free rpi
                                                  \PYGZca{} OS version
\end{sphinxVerbatim}
\begin{description}
\item[{\sphinxcode{\sphinxupquote{apt-get update}} (optionnel)}] \leavevmode
pour mettre notre cache à jour si ce n’est pas déjà fait

\item[{\sphinxcode{\sphinxupquote{apt-cache search monpaquet}} (optionnel)}] \leavevmode
pour rechercher le paquet que nous voulons télécharger si nous ne connaissons pas son nom exact

\item[{\sphinxcode{\sphinxupquote{apt-get install monpaquet}}}] \leavevmode
pour télécharger et installer notre paquet

\item[{\sphinxcode{\sphinxupquote{apt-cache show nomdupaquet}}}] \leavevmode
pour une plus ample description d’un paquet

\item[{\sphinxcode{\sphinxupquote{apt-get remove nomdupaquet}}}] \leavevmode
désinstaller un paquet. Toutefois, cela ne supprime pas les dépendances du paquet devenues inutiles

\item[{\sphinxcode{\sphinxupquote{apt-get autoremove nomdupaquet}}}] \leavevmode
Pour demander à \sphinxcode{\sphinxupquote{apt-get}} de supprimer aussi les dépendances inutiles, on utilise \sphinxcode{\sphinxupquote{autoremove}}

\item[{\sphinxcode{\sphinxupquote{apt-get upgrade}}}] \leavevmode
met à jour tous les paquets du système d’un seul couper (faire un \sphinxcode{\sphinxupquote{apt-get update}} avant)

\end{description}


\chapter{A propos des manuels}
\label{\detokenize{13-man:a-propos-des-manuels}}\label{\detokenize{13-man::doc}}\begin{description}
\item[{\sphinxcode{\sphinxupquote{man}}}] \leavevmode\begin{itemize}
\item {} 
\sphinxcode{\sphinxupquote{/ + "..." + Entrée}} : faire une recherche. Résultat suivant : à nouveau taper /

\end{itemize}

\end{description}

synopsis dans le man :
\begin{itemize}
\item {} 
gras : tapez le mot exactement comme indiqué

\item {} 
souligné : remplacez le mot souligné par la valeur qui convient dans votre cas

\item {} 
\sphinxcode{\sphinxupquote{{[}-hvc{]}}} : toutes les options -h, -v et -c sont facultatives

\item {} 
\sphinxcode{\sphinxupquote{a\textbar{}b}} : vous pouvez écrire l’option « a » OU « b », mais pas les deux à la fois

\item {} 
\sphinxcode{\sphinxupquote{option...}} : les points de suspension indiquent que l’option peut être répétée autant de fois que vous voulez

\end{itemize}
\begin{description}
\item[{\sphinxcode{\sphinxupquote{apropos quelqueChose}}}] \leavevmode
va rechercher toutes les commandes qui parlent de quelqueChose dans leur page du manuel.

\item[{\sphinxcode{\sphinxupquote{cmd -h}}}] \leavevmode
si implémenté par la commande, affiche une aide résumée

\item[{\sphinxcode{\sphinxupquote{whatis cmd}}}] \leavevmode
donne juste l’en-tête du manuel pour expliquer en deux mots à quoi sert la commande

\end{description}


\chapter{Rechercher des fichiers}
\label{\detokenize{14-rechercher:rechercher-des-fichiers}}\label{\detokenize{14-rechercher::doc}}\begin{description}
\item[{\sphinxcode{\sphinxupquote{locate fichier}}}] \leavevmode
recherche les fichiers/dossiers qui contiennent « fichier » et qui sont dans la base de données (db)

\item[{\sphinxcode{\sphinxupquote{sudo updatedb}}}] \leavevmode
mettre à jour la base de données

\item[{\sphinxcode{\sphinxupquote{find fichier}}}] \leavevmode
recherche des fichiers en scannant le disque dur (pas la db base)

\item[{\sphinxcode{\sphinxupquote{find -name "file.txt"}}}] \leavevmode
dans le dossier courant

\item[{\sphinxcode{\sphinxupquote{find /var/log/ -name "syslog"}}}] \leavevmode
dans un dossier différent du courant

\item[{\sphinxcode{\sphinxupquote{find -size +10M}}}] \leavevmode
chercher les fichiers qui font plus de 10Mo

\item[{\sphinxcode{\sphinxupquote{find -name "*.odt" -atime -6}}}] \leavevmode
chercher les fichiers .odt auxquels on a accédé depuis 7 jours (numérotation comence à 0)

\item[{\sphinxcode{\sphinxupquote{find -type d}}}] \leavevmode
only directories

\item[{\sphinxcode{\sphinxupquote{find -type f}}}] \leavevmode
only files

\item[{\sphinxcode{\sphinxupquote{find -name "*.jpg" -delete}}}] \leavevmode
supprime tous les fichiers .jpg (dangereux !!!)

\item[{\sphinxcode{\sphinxupquote{find -name ".jpg" -exec cmd \{\} \textbackslash{};}}}] \leavevmode
exécute la commande cmd avec les résultats. La commande doit finir par un \sphinxcode{\sphinxupquote{\textbackslash{};}} obligatoirement

\end{description}


\chapter{Archiver et compresser}
\label{\detokenize{15-archiver:archiver-et-compresser}}\label{\detokenize{15-archiver::doc}}
fichier \sphinxcode{\sphinxupquote{tar}} : archive permettant de regrouper plusieurs fichiers
\begin{description}
\item[{\sphinxcode{\sphinxupquote{gzip}}/\sphinxcode{\sphinxupquote{bzip2}}}] \leavevmode{[}pour compresser des fichiers (des tar par exemple){]}\begin{itemize}
\item {} 
\sphinxcode{\sphinxupquote{gzip}} : c’est le plus connu et le plus utilisé ;

\item {} 
\sphinxcode{\sphinxupquote{bzip2}} : il est un peu moins fréquemment utilisé. Il compresse mieux mais plus lentement que \sphinxcode{\sphinxupquote{gzip}}.

\end{itemize}

\item[{\sphinxcode{\sphinxupquote{tar -cvf nom\_archive.tar nom\_dossier/}}}] \leavevmode
créer une archive
\begin{itemize}
\item {} 
\sphinxcode{\sphinxupquote{-c}} : signifie créer une archive tar

\item {} 
\sphinxcode{\sphinxupquote{-v}} : signifie afficher le détail des opérations

\item {} 
\sphinxcode{\sphinxupquote{-f}} : signifie assembler l’archive dans un fichier.

\end{itemize}

\end{description}


\section{Archiver}
\label{\detokenize{15-archiver:archiver}}\begin{description}
\item[{\sphinxcode{\sphinxupquote{tar -tf archive.tar}}}] \leavevmode
afficher le contenu de l’archive sans l’extraire

\item[{\sphinxcode{\sphinxupquote{tar -rvf archive.tar fichier\_supplementaire}}}] \leavevmode
ajouter un fichier à l’archive

\item[{\sphinxcode{\sphinxupquote{tar -xvf archive.tar}}}] \leavevmode
extraire les fichiers de l’archive

\end{description}


\section{Compresser une archive}
\label{\detokenize{15-archiver:compresser-une-archive}}\begin{description}
\item[{\sphinxcode{\sphinxupquote{gzip archive.tar}}}] \leavevmode
compresse l’archive avec gzip -\textgreater{} ajoute un .gz à la fin (archive.tar.gz)

\item[{\sphinxcode{\sphinxupquote{gunzip archive.tar.gz}}}] \leavevmode
décompresse l’archive

\item[{\sphinxcode{\sphinxupquote{bzip2 archive.tar}}}] \leavevmode
idem avec \sphinxcode{\sphinxupquote{bzip2}}

\item[{\sphinxcode{\sphinxupquote{bunzip2 archive.tar.bz2}}}] \leavevmode
idem avec \sphinxcode{\sphinxupquote{bzip2}}

\end{description}


\section{Archiver et compresser en une commande}
\label{\detokenize{15-archiver:archiver-et-compresser-en-une-commande}}\begin{description}
\item[{\sphinxcode{\sphinxupquote{tar -zcvf archive.tar.gz tutoriels/}}}] \leavevmode
archive et compresse en une comande (ici avec \sphinxcode{\sphinxupquote{gzip}})

\item[{\sphinxcode{\sphinxupquote{tar -zxvf archive.tar.gz}}}] \leavevmode
décompresse et désarchive en une commande

\item[{\sphinxcode{\sphinxupquote{tar -jcvf tutoriels.tar.bz2 tutoriels/}}}] \leavevmode
idem avec \sphinxcode{\sphinxupquote{bzip2}}

\item[{\sphinxcode{\sphinxupquote{tar -jxvf tutoriels.tar.bz2 tutoriels/}}}] \leavevmode
idem avec \sphinxcode{\sphinxupquote{bzip2}}

\end{description}

\sphinxcode{\sphinxupquote{tar -jxvf tutoriels.tar.bz2 tutoriels/}}
\begin{quote}

idem avec \sphinxcode{\sphinxupquote{bzip2}}
\end{quote}


\section{Afficher un fichier archivé sans le désarchiver}
\label{\detokenize{15-archiver:afficher-un-fichier-archive-sans-le-desarchiver}}
\sphinxcode{\sphinxupquote{zcat}}, \sphinxcode{\sphinxupquote{zmore}} \& \sphinxcode{\sphinxupquote{zless}}
\begin{quote}

afficher directement un fichier compressé (fichier simple, pas archive)
\end{quote}


\section{Fichiers \sphinxstyleliteralintitle{\sphinxupquote{.zip}}}
\label{\detokenize{15-archiver:fichiers-zip}}
\sphinxcode{\sphinxupquote{sudo apt-get install unzip}}
\begin{quote}

installer le décompresseur de zip
\end{quote}

\sphinxcode{\sphinxupquote{unzip archive.zip}}
\begin{quote}

décompresser un zip
\end{quote}

\sphinxcode{\sphinxupquote{unzip -l fichier.zip}}
\begin{quote}

afficher le contenu du fichier zip sans l’extraire
\end{quote}

\sphinxcode{\sphinxupquote{sudo apt-get install zip}}
\begin{quote}

installer le compresseur de zip
\end{quote}

\sphinxcode{\sphinxupquote{zip -r tutoriels.zip tutoriels/}}
\begin{quote}

Le -r demande à compresser tous les fichiers contenus dans le dossier tutoriels (sans ce paramètre, seul le dossier, vide, sera compressé !).
\end{quote}


\section{Fichiers \sphinxstyleliteralintitle{\sphinxupquote{.rar}}}
\label{\detokenize{15-archiver:fichiers-rar}}
\sphinxcode{\sphinxupquote{sudo apt-get install unrar}}

\sphinxcode{\sphinxupquote{unrar e tutoriels.rar}}
{}`   Non, vous ne rêvez pas, l’auteur du programme ne veut pas que l’on mette un tiret devant l’option e ! Il faut bien qu’il y ait des exceptions dans la vie. :-)
\begin{description}
\item[{\sphinxcode{\sphinxupquote{unrar l tutoriels.rar}}}] \leavevmode
Pour lister le contenu avant décompression, utilisez l’option
\begin{quote}

\sphinxstylestrong{pas possible de créer des fichiers .rar (format propriétaire)}
\end{quote}

\end{description}


\chapter{La connexion sécurisée à distance avec SSH}
\label{\detokenize{16-ssh:la-connexion-securisee-a-distance-avec-ssh}}\label{\detokenize{16-ssh::doc}}\begin{itemize}
\item {} 
Telnet : non sécurisé (non crypté)

\item {} 
SSH : \textasciitilde{}Telnet crypté

\end{itemize}
\begin{description}
\item[{\sphinxcode{\sphinxupquote{ssh rico@192.168.1.5}}}] \leavevmode
se connecte en ssh au login \sphinxcode{\sphinxupquote{rico}} et à la machine d’adresse ip \sphinxcode{\sphinxupquote{192.168.1.5}}

\end{description}


\chapter{Transférer des fichiers}
\label{\detokenize{17-transferer-fichiers:transferer-des-fichiers}}\label{\detokenize{17-transferer-fichiers::doc}}

\section{\sphinxstyleliteralintitle{\sphinxupquote{wget}} : télécharger un fichier sur le web}
\label{\detokenize{17-transferer-fichiers:wget-telecharger-un-fichier-sur-le-web}}\begin{description}
\item[{\sphinxcode{\sphinxupquote{wget http://website.com/file}}}] \leavevmode
télécharger un fichier distant

\end{description}


\section{\sphinxstyleliteralintitle{\sphinxupquote{scp}} : copier/coller un fichier par SSH}
\label{\detokenize{17-transferer-fichiers:scp-copier-coller-un-fichier-par-ssh}}\begin{description}
\item[{\sphinxcode{\sphinxupquote{scp fichier\_origine copie\_destination}}}] \leavevmode
permet de copier des fichiers distants de manière sécurisée (cryptage ssh)

\item[{\sphinxcode{\sphinxupquote{scp fichier remoteLogin@85.123.10.201:/home/remoteLogin/dossier/}}}] \leavevmode
copier un fichier de l’ordi local vers le distant

\item[{\sphinxcode{\sphinxupquote{scp remoteLogin@85.123.10.201:fichier copie\_fichier}}}] \leavevmode
pas nécessaire de préciser le nom -\textgreater{} gardera le même nom que le fichier d’origine

\item[{\sphinxcode{\sphinxupquote{scp -P 16296 mateo21@85.123.10.201:image.png}}}] \leavevmode
en précisant le port (attention MAJUSCULE)

\end{description}


\section{Protocole ftp non sécurisé}
\label{\detokenize{17-transferer-fichiers:protocole-ftp-non-securise}}\begin{description}
\item[{\sphinxcode{\sphinxupquote{ftp://ftp.debian.org}}}] \leavevmode
connection au serveur ftp de debian (login : anonymous mpd : any)
\begin{itemize}
\item {} 
on a ensuite un prompt qui nous permet de naviguer sur le serveur (\sphinxcode{\sphinxupquote{ls}}, \sphinxcode{\sphinxupquote{cd}}, \sphinxcode{\sphinxupquote{pwd}}…)

\item {} 
\sphinxcode{\sphinxupquote{put}} : ajouter un fichier sur le serveur (verouillé dans le cas de celui de debian)

\item {} 
\sphinxcode{\sphinxupquote{get}} : récupérer un fichier depuis le serveur (sera mis dans le dossier courant du pc local)

\item {} 
pour se déplacer dans le pc local : \sphinxcode{\sphinxupquote{!cd}}, \sphinxcode{\sphinxupquote{!ls}}, \sphinxcode{\sphinxupquote{!pwd}}… (ajouter un ! avant la commande)

\item {} 
attention : protocole ftp pas sécurisée

\end{itemize}

\end{description}


\section{Protocole ftp sécurisé}
\label{\detokenize{17-transferer-fichiers:protocole-ftp-securise}}\begin{description}
\item[{\sphinxcode{\sphinxupquote{sftp mateo21@lisa.simple-it.fr}}}] \leavevmode
ftp sécurisée avec ssh (port par défaut : 22)

\end{description}


\section{\sphinxstyleliteralintitle{\sphinxupquote{rsync}} : sauvegardes sur un serveur distant}
\label{\detokenize{17-transferer-fichiers:rsync-sauvegardes-sur-un-serveur-distant}}\begin{description}
\item[{\sphinxcode{\sphinxupquote{rsync}}}] \leavevmode
permet de créer des sauvegardes sur un serveur distant (incrémentielles, etc…)

\item[{\sphinxcode{\sphinxupquote{rsync -arv Images/ backups/}}}] \leavevmode
analyse les différences entre /Images et /backup et fait une sauvegarde
\begin{itemize}
\item {} 
\sphinxcode{\sphinxupquote{-a}} : conserve toutes les informations sur les fichiers, comme les droits (chmod), la date de modification, etc. ;

\item {} 
\sphinxcode{\sphinxupquote{-r}} : sauvegarde aussi tous les sous-dossiers qui se trouvent dans le dossier à sauvegarder ;

\item {} 
\sphinxcode{\sphinxupquote{-v}} : mode verbeux, affiche des informations détaillées sur la copie en cours.

\end{itemize}

\item[{\sphinxcode{\sphinxupquote{rsync -arv -{-}delete Images/ backups/}}}] \leavevmode
analyse les différences entre /Images et /backup et efface les fichiers de backups qui ne sont plus dans /Images

\item[{\sphinxcode{\sphinxupquote{rsync -arv -{-}delete -{-}backup Images/ backups/}}}] \leavevmode
garde les fichiers suprimés en leur ajoutant un suffixe dans le dossier de sauvegarde

\item[{\sphinxcode{\sphinxupquote{rsync -arv -{-}delete -{-}backup -{-}backup-dir=/home/mateo21/backups\_supprimes Images/ backups/}}}] \leavevmode
les fichiers suprimés vont dans le dossier /home/mateo21/backups\_supprimes

\item[{\sphinxcode{\sphinxupquote{rsync -arv -{-}delete -{-}backup -{-}backup-dir=/home/mateo21/fichiers\_supprimes Images/ mateo21@IP\_du\_serveur:mes\_backups/}}}] \leavevmode
fait le backup sur un ordinateur distant via ssh

\item[{\sphinxcode{\sphinxupquote{rsync -arv -{-}delete -{-}backup -{-}backup-dir=/home/mateo21/fichiers\_supprimes Images/ mateo21@IP\_du\_serveur:mes\_backups/ -e "ssh -p 12473"}}}] \leavevmode
avec un n° de port custom

\end{description}


\chapter{Analyser le réseau et filtrer le trafic avec un pare-feu}
\label{\detokenize{18-pare-feu:analyser-le-reseau-et-filtrer-le-trafic-avec-un-pare-feu}}\label{\detokenize{18-pare-feu::doc}}\begin{description}
\item[{\sphinxcode{\sphinxupquote{host www.google.com}}}] \leavevmode
renvoie l’adresse ip du site

\item[{pour voir les associations (\textasciitilde{}DNS) locales :}] \leavevmode
-\textgreater{} /etc/hosts

\item[{\sphinxcode{\sphinxupquote{whois google.com}}}] \leavevmode
obtenir des infos sur un site (installer d’abord le package whois)

\item[{\sphinxcode{\sphinxupquote{ifconfig}}}] \leavevmode
liste des interfaces réseau
\sphinxcode{\sphinxupquote{lo}} : c’est la boucle locale. Tout le monde devrait avoir cette interface. Elle correspond à une connexion à… vous-mêmes. C’est pour cela qu’on l’appelle la boucle locale : tout ce qui est envoyé par là vous revient automatiquement. Cela peut paraître inutile, mais on a parfois besoin de se connecter à soi-même pour des raisons pratiques.

\item[{\sphinxcode{\sphinxupquote{ifconfig eth0 down/up}}}] \leavevmode
activer/désactiver une interface

\item[{\sphinxcode{\sphinxupquote{netstat}}}] \leavevmode
statistiques sur le réseau

\item[{\sphinxcode{\sphinxupquote{netstat -i}}}] \leavevmode
statistiques des interfaces réseau

\item[{\sphinxcode{\sphinxupquote{netstat -uta}}}] \leavevmode
lister toutes les connexions ouvertes

Options :
\begin{itemize}
\item {} 
\sphinxcode{\sphinxupquote{-u}} : afficher les connexions UDP ;

\item {} 
\sphinxcode{\sphinxupquote{-t}} : afficher les connexions TCP ;

\item {} 
\sphinxcode{\sphinxupquote{-a}} : afficher toutes les connexions quel que soit leur état.

\item {} 
\sphinxcode{\sphinxupquote{-n}} : affiche les n° de port plutôt que leur description

\item {} 
\sphinxcode{\sphinxupquote{-l}} : affiche les connexions en état d’écoute

\end{itemize}

Statuts :
\begin{itemize}
\item {} 
\sphinxcode{\sphinxupquote{ESTABLISHED}} : la connexion a été établie avec l’ordinateur distant

\item {} 
\sphinxcode{\sphinxupquote{TIME\_WAIT}} : la connexion attend le traitement de tous les paquets encore sur le réseau avant de commencer la fermeture

\item {} 
\sphinxcode{\sphinxupquote{CLOSE\_WAIT}} : le serveur distant a arrêté la connexion de lui-même (peut-être parce que vous êtes restés inactifs trop longtemps ?)

\item {} 
\sphinxcode{\sphinxupquote{CLOSED}} : la connexion n’est pas utilisée

\item {} 
\sphinxcode{\sphinxupquote{CLOSING}} : la fermeture de la connexion est entamée mais toutes les données n’ont pas encore été envoyées

\item {} 
\sphinxcode{\sphinxupquote{LISTEN}} : à l’écoute des connexions entrantes. Les connexions à l’état LISTEN ne sont pas utilisées actuellement mais qu’elles « écoutent » le réseau au cas où quelqu’un veuille se connecter à votre ordinateur

\end{itemize}

\item[{\sphinxcode{\sphinxupquote{netstat -s}}}] \leavevmode
statistiques résumées

\item[{\sphinxcode{\sphinxupquote{iptables -L}}}] \leavevmode
afficher les règles
\begin{itemize}
\item {} 
\sphinxcode{\sphinxupquote{Chain INPUT}} : correspond aux règles manipulant le trafic entrant ;

\item {} 
\sphinxcode{\sphinxupquote{Chain FORWARD}} : correspond aux règles manipulant la redirection du trafic ;

\item {} 
\sphinxcode{\sphinxupquote{Chain OUTPUT}} : correspond aux règles manipulant le trafic sortant.

\item {} 
\sphinxcode{\sphinxupquote{target}} : ce que fait la règle. Ici c’est ACCEPT, c’est-à-dire que cette ligne autorise un port et / ou une IP

\item {} 
\sphinxcode{\sphinxupquote{prot}} : le protocole utilisé (tcp, udp, icmp). Je rappelle que TCP est celui auquel on a le plus recourt. ICMP permet à votre ordinateur de répondre aux requêtes de type « ping »

\item {} 
\sphinxcode{\sphinxupquote{source}} : l’IP de source. Pour INPUT, la source est l’ordinateur distant qui se connecte à vous

\item {} 
\sphinxcode{\sphinxupquote{destination}} : l’IP de destination. Pour OUTPUT, c’est l’ordinateur auquel on se connecte

\item {} 
la dernière colonne : elle indique le port après les deux points « : ». Ce port est affiché en toutes lettres, mais avec -n vous pouvez obtenir le numéro correspondant

\end{itemize}

\item[{\sphinxcode{\sphinxupquote{iptables -F}}}] \leavevmode
réinitialise les règles du pare-feu (attention : efface tout le travail fait auparavant sur le pare-feu…)

\end{description}

Ajouter et supprimer des règles :
\begin{quote}

Voici les principales commandes à connaître :
\begin{itemize}
\item {} 
\sphinxcode{\sphinxupquote{-A chain}} : ajoute une règle en fin de liste pour la chain indiquée (INPUT ou OUTPUT, par exemple).

\item {} 
\sphinxcode{\sphinxupquote{-D chain rulenum}} : supprime la règle n° rulenum pour la chain indiquée.

\item {} 
\sphinxcode{\sphinxupquote{-I chain rulenum}} : insère une règle au milieu de la liste à la position indiquée par rulenum. Si vous n’indiquez pas de position rulenum, la règle sera insérée en premier, tout en haut dans la liste.

\item {} 
\sphinxcode{\sphinxupquote{-R chain rulenum}} : remplace la règle n° rulenum dans la chain indiquée.

\item {} 
\sphinxcode{\sphinxupquote{-L}} : liste les règles (nous l’avons déjà vu).

\item {} 
\sphinxcode{\sphinxupquote{-F chain}} : vide toutes les règles de la chain indiquée. Cela revient à supprimer toutes les règles une par une pour cette chain.

\item {} 
\sphinxcode{\sphinxupquote{-P chain regle}} : modifie la règle par défaut pour la chain. Cela permet de dire, par exemple, que par défaut tous les ports sont fermés, sauf ceux que l’on a indiqués dans les règles.

\item {} 
\sphinxcode{\sphinxupquote{-m (-{-}match)}} : Specifies  a  match  to use, that is, an extension module that tests for a specific property. The set of matches make up the condition under which a target is invoked. Matches are evaluated first to last as specified on the command line and work in short-circuit fashion,  i.e.  if one extension yields false, evaluation will stop.

\end{itemize}
\end{quote}
\begin{description}
\item[{\sphinxcode{\sphinxupquote{iptables -A (chain) -p (protocole) -{-}dport (port) -j (décision)}}}] \leavevmode
ajouter une règle. Remplacez chain par la section qui vous intéresse (\sphinxcode{\sphinxupquote{INPUT}} ou \sphinxcode{\sphinxupquote{OUTPUT}}), protocole par le nom du protocole à filtrer (TCP, UDP, ICMP…) et enfin décision par la décision à prendre : \sphinxcode{\sphinxupquote{ACCEPT}} pour accepter le paquet, \sphinxcode{\sphinxupquote{REJECT}} pour le rejeter ou bien \sphinxcode{\sphinxupquote{DROP}} pour l’ignorer complètement.

\end{description}

exemple : \sphinxcode{\sphinxupquote{iptables -A INPUT -p tcp -{-}dport ssh -j ACCEPT}}
\begin{description}
\item[{\sphinxcode{\sphinxupquote{iptables -A INPUT -p icmp -j ACCEPT}}}] \leavevmode
autoriser les pings (protocole ICMP)

\end{description}

\sphinxcode{\sphinxupquote{iptables -A INPUT -i lo -j ACCEPT}}

\sphinxcode{\sphinxupquote{iptables -A INPUT -m state -{-}state ESTABLISHED,RELATED -j ACCEPT}}
\begin{itemize}
\item {} 
La première règle autorise tout le trafic sur l’interface de loopback locale grâce à -i lo. Il n’y a pas de risque à autoriser votre ordinateur à communiquer avec lui-même, d’autant plus qu’il en a parfois besoin !

\item {} 
La seconde règle autorise toutes les connexions qui sont déjà à l’état ESTABLISHED ou RELATED. En clair, elle autorise toutes les connexions qui ont été demandées par votre PC. Là encore, cela permet d’assouplir le pare-feu et de le rendre fonctionnel pour une utilisation quotidienne.

\end{itemize}
\begin{description}
\item[{\sphinxcode{\sphinxupquote{iptables -P INPUT DROP}}}] \leavevmode
on refuse tout ce qui n’est pas aurtorisé (faire de même pour OUTPUT)

\end{description}

ATTENTION : ces règles seront perdues au redémarrage ! pour que ça ne soit pas le cas -\textgreater{} ajouter un script shell


\chapter{Compiler un programme depuis les sources}
\label{\detokenize{19-compiler:compiler-un-programme-depuis-les-sources}}\label{\detokenize{19-compiler::doc}}
fichier \sphinxcode{\sphinxupquote{.deb}} : pour installer un programme sur les dérivées de Debian (Red Hat : \sphinxcode{\sphinxupquote{.rpm}})

Convertir un \sphinxcode{\sphinxupquote{.rpm}} en \sphinxcode{\sphinxupquote{.deb}} : utiliser le programme \sphinxcode{\sphinxupquote{alien}}
\begin{description}
\item[{\sphinxcode{\sphinxupquote{dpkg -i foo.deb}}}] \leavevmode
installer un package \sphinxcode{\sphinxupquote{foo.deb}}

\end{description}


\section{Pour compiler un programme depuis une source}
\label{\detokenize{19-compiler:pour-compiler-un-programme-depuis-une-source}}\begin{enumerate}
\def\theenumi{\arabic{enumi}}
\def\labelenumi{\theenumi .}
\makeatletter\def\p@enumii{\p@enumi \theenumi .}\makeatother
\item {} \begin{description}
\item[{Installer build-ensential (si pas encore installé)}] \leavevmode
\sphinxcode{\sphinxupquote{sudo apt-get install build-essential}}

\end{description}

\item {} \begin{description}
\item[{Récupérer l’archive contenant le programme et la décompresse/désarchiver :}] \leavevmode
\sphinxcode{\sphinxupquote{tar zxvf htop-0.8.3.tar.gz}} (ici \sphinxcode{\sphinxupquote{htop-0.8.3}})

\end{description}

\item {} \begin{description}
\item[{Exécuter \sphinxcode{\sphinxupquote{./configure}} pour vérifier les dépendances}] \leavevmode
\sphinxcode{\sphinxupquote{./configure}}

\end{description}

\item {} 
Installer ce qui manque au fur et à mesure (chercher sur google pour trouver le nom du paquet à installer pour les headers manquants par exemple) et relancer \sphinxcode{\sphinxupquote{./configure}} jusqu’à ce qu’il n’y ait plus d’erreurs

\item {} \begin{description}
\item[{Lancer la compilation}] \leavevmode
\sphinxcode{\sphinxupquote{make}} (attention à être dans le répertoire de la source)

\end{description}

\item {} \begin{description}
\item[{A la fin de la compilation un éxécutable a été créé. Il ne reste plus qu’à l’installer, c’est-à-dire à le copier dans le bon répertoire. Là encore, vous n’avez pas à vous poser beaucoup de questions. Exécutez la commande suivante :}] \leavevmode
\sphinxcode{\sphinxupquote{sudo make install}}

\end{description}

\item {} \begin{description}
\item[{Le programme est installé ! Pour le lancer :}] \leavevmode
\sphinxcode{\sphinxupquote{htop}} (dans le cas de notre exemple)

\end{description}

\item {} 
On peut maintenant supprimer le répertoire contenant les fichiers source

\end{enumerate}


\chapter{VIM : éditeur de texte avancé}
\label{\detokenize{20-vim:vim-editeur-de-texte-avance}}\label{\detokenize{20-vim::doc}}\begin{description}
\item[{\sphinxcode{\sphinxupquote{vim}}}] \leavevmode
lancer vim

\item[{\sphinxcode{\sphinxupquote{vimtutor}}}] \leavevmode
lancer le tutoriel vim

\end{description}


\section{Modes de VIM}
\label{\detokenize{20-vim:modes-de-vim}}\begin{itemize}
\item {} 
\sphinxstylestrong{Mode interactif} : par défaut. Permet de se déplacer dans le texte, de supprimer une ligne, copier-coller du texte, rejoindre une ligne précise, annuler ses actions, etc

\item {} 
\sphinxstylestrong{Mode insertion} : appuyer sur \sphinxcode{\sphinxupquote{i}} pour y entrer

\item {} 
\sphinxstylestrong{Mode commande} : l’activer en tapant \sphinxcode{\sphinxupquote{:}}

\end{itemize}


\begin{savenotes}\sphinxattablestart
\centering
\begin{tabular}[t]{|*{2}{\X{1}{2}|}}
\hline
\sphinxstartmulticolumn{2}%
\begin{varwidth}[t]{\sphinxcolwidth{2}{2}}
\sphinxstyletheadfamily Racourcis
\par
\vskip-\baselineskip\vbox{\hbox{\strut}}\end{varwidth}%
\sphinxstopmulticolumn
\\
\hline
\sphinxcode{\sphinxupquote{i}}
&
\begin{DUlineblock}{0em}
\item[] insérer du texte
\end{DUlineblock}
\\
\hline
\sphinxcode{\sphinxupquote{ESC}}
&
\begin{DUlineblock}{0em}
\item[] sortir du mode insertion
\end{DUlineblock}
\\
\hline
\end{tabular}
\par
\sphinxattableend\end{savenotes}

\begin{DUlineblock}{0em}
\item[] 
\end{DUlineblock}


\begin{savenotes}\sphinxattablestart
\centering
\begin{tabular}[t]{|*{2}{\X{1}{2}|}}
\hline
\sphinxstartmulticolumn{2}%
\begin{varwidth}[t]{\sphinxcolwidth{2}{2}}
\sphinxstyletheadfamily Commandes
\par
\vskip-\baselineskip\vbox{\hbox{\strut}}\end{varwidth}%
\sphinxstopmulticolumn
\\
\hline
\sphinxcode{\sphinxupquote{:w}}
&
\begin{DUlineblock}{0em}
\item[] enregistrer le fichier (write)
\end{DUlineblock}
\\
\hline
\sphinxcode{\sphinxupquote{:q}}
&
\begin{DUlineblock}{0em}
\item[] quitter
\end{DUlineblock}
\\
\hline
\sphinxcode{\sphinxupquote{:wq}}
&
\begin{DUlineblock}{0em}
\item[] enregistrer puis quitter
\end{DUlineblock}
\\
\hline
\end{tabular}
\par
\sphinxattableend\end{savenotes}

\begin{DUlineblock}{0em}
\item[] 
\end{DUlineblock}


\begin{savenotes}\sphinxattablestart
\centering
\begin{tabular}[t]{|*{2}{\X{1}{2}|}}
\hline
\sphinxstartmulticolumn{2}%
\begin{varwidth}[t]{\sphinxcolwidth{2}{2}}
\sphinxstyletheadfamily Déplacements
\par
\vskip-\baselineskip\vbox{\hbox{\strut}}\end{varwidth}%
\sphinxstopmulticolumn
\\
\hline
\sphinxcode{\sphinxupquote{h}}
&
\begin{DUlineblock}{0em}
\item[] gauche
\end{DUlineblock}
\\
\hline
\sphinxcode{\sphinxupquote{j}}
&
\begin{DUlineblock}{0em}
\item[] bas
\end{DUlineblock}
\\
\hline
\sphinxcode{\sphinxupquote{k}}
&
\begin{DUlineblock}{0em}
\item[] haut
\end{DUlineblock}
\\
\hline
\sphinxcode{\sphinxupquote{l}}
&
\begin{DUlineblock}{0em}
\item[] droite
\end{DUlineblock}
\\
\hline
\sphinxcode{\sphinxupquote{0}}
&
\begin{DUlineblock}{0em}
\item[] aller en début de ligne (origin)
\end{DUlineblock}
\\
\hline
\sphinxcode{\sphinxupquote{\$}}
&
\begin{DUlineblock}{0em}
\item[] aller en fin de ligne
\end{DUlineblock}
\\
\hline
\sphinxcode{\sphinxupquote{w}}
&
\begin{DUlineblock}{0em}
\item[] se déplacer de mot en mot (word)
\end{DUlineblock}
\\
\hline
\end{tabular}
\par
\sphinxattableend\end{savenotes}

\fvset{hllines={, ,}}%
\begin{sphinxVerbatim}[commandchars=\\\{\}]
      \PYGZca{}
      k
\PYGZlt{} h       l \PYGZgt{}
      j
      v
\end{sphinxVerbatim}

\begin{DUlineblock}{0em}
\item[] 
\end{DUlineblock}


\begin{savenotes}\sphinxattablestart
\centering
\begin{tabular}[t]{|*{2}{\X{1}{2}|}}
\hline
\sphinxstartmulticolumn{2}%
\begin{varwidth}[t]{\sphinxcolwidth{2}{2}}
\sphinxstyletheadfamily Opérations standards
\par
\vskip-\baselineskip\vbox{\hbox{\strut}}\end{varwidth}%
\sphinxstopmulticolumn
\\
\hline
\sphinxcode{\sphinxupquote{x}}
&
\begin{DUlineblock}{0em}
\item[] effacer des lettres en mode interactif
\end{DUlineblock}
\\
\hline
\sphinxcode{\sphinxupquote{(nb)x}}
&
\begin{DUlineblock}{0em}
\item[] effacer nb lettres
\end{DUlineblock}
\\
\hline
\sphinxcode{\sphinxupquote{dw}}
&
\begin{DUlineblock}{0em}
\item[] effacer un mot
\end{DUlineblock}
\\
\hline
\sphinxcode{\sphinxupquote{dd}}
&
\begin{DUlineblock}{0em}
\item[] effacer une ligne
\end{DUlineblock}
\\
\hline
\sphinxcode{\sphinxupquote{d0}}
&
\begin{DUlineblock}{0em}
\item[] supprimer du curseur au début de la ligne
\end{DUlineblock}
\\
\hline
\sphinxcode{\sphinxupquote{d\$}}
&
\begin{DUlineblock}{0em}
\item[] supprimer du curseur à la fin de la ligne
\end{DUlineblock}
\\
\hline
\sphinxcode{\sphinxupquote{yy}}
&
\begin{DUlineblock}{0em}
\item[] copier une ligne en mémoire
\end{DUlineblock}
\\
\hline
\sphinxcode{\sphinxupquote{p}}
&
\begin{DUlineblock}{0em}
\item[] coller (coller plusieurs fois : ex : 8p -\textgreater{} 8x)
\end{DUlineblock}
\\
\hline
\sphinxcode{\sphinxupquote{r}}
&
\begin{DUlineblock}{0em}
\item[] remplacer une lettre
\end{DUlineblock}
\\
\hline
\sphinxcode{\sphinxupquote{u}}
&
\begin{DUlineblock}{0em}
\item[] annuler des modifications
\end{DUlineblock}
\\
\hline
\sphinxcode{\sphinxupquote{G}}
&
\begin{DUlineblock}{0em}
\item[] aller à la ligne x (Go)
\end{DUlineblock}
\\
\hline
\end{tabular}
\par
\sphinxattableend\end{savenotes}

\begin{DUlineblock}{0em}
\item[] 
\end{DUlineblock}


\begin{savenotes}\sphinxattablestart
\centering
\begin{tabular}[t]{|*{2}{\X{1}{2}|}}
\hline
\sphinxstartmulticolumn{2}%
\begin{varwidth}[t]{\sphinxcolwidth{2}{2}}
\sphinxstyletheadfamily Opérations avancées
\par
\vskip-\baselineskip\vbox{\hbox{\strut}}\end{varwidth}%
\sphinxstopmulticolumn
\\
\hline
\sphinxcode{\sphinxupquote{/}}
&
\begin{DUlineblock}{0em}
\item[] passer en mode recherche (pour chercher un mot par ex)
\item[] \sphinxcode{\sphinxupquote{n}}   aller à la prochaine occurence
\item[] \sphinxcode{\sphinxupquote{N}}   aller à la précédente occurence
\end{DUlineblock}
\\
\hline
\sphinxcode{\sphinxupquote{:s/ancien/nouveau}}
&
\begin{DUlineblock}{0em}
\item[] remplacer le mot « ancien » par le mot « nouveau » :
\end{DUlineblock}
\\
\hline
\sphinxcode{\sphinxupquote{:s/ancien/nouveau}}
&
\begin{DUlineblock}{0em}
\item[] remplace la première occurrence de la ligne où se trouve le curseur ;
\end{DUlineblock}
\\
\hline
\sphinxcode{\sphinxupquote{:s/ancien/nouveau/g}}
&
\begin{DUlineblock}{0em}
\item[] remplace toutes les occurrences de la ligne où se trouve le curseur ;
\end{DUlineblock}
\\
\hline
\sphinxcode{\sphinxupquote{:\#,\#s/ancien/nouveau/g}}
&
\begin{DUlineblock}{0em}
\item[] remplace toutes les occurrences dans les lignes n° \# à \# du fichier ;
\end{DUlineblock}
\\
\hline
\sphinxcode{\sphinxupquote{:\%s/ancien/nouveau/g}}
&
\begin{DUlineblock}{0em}
\item[] remplace toutes les occurrences dans tout le fichier. C’est peut-être ce que vous utiliserez le plus fréquemment.
\end{DUlineblock}
\\
\hline
\sphinxcode{\sphinxupquote{:r}}
&
\begin{DUlineblock}{0em}
\item[] fusion de fichiers : insérer le contenu d’un fichier au curseur
\end{DUlineblock}
\\
\hline
\end{tabular}
\par
\sphinxattableend\end{savenotes}

\begin{DUlineblock}{0em}
\item[] 
\end{DUlineblock}


\begin{savenotes}\sphinxattablestart
\centering
\begin{tabular}[t]{|*{2}{\X{1}{2}|}}
\hline
\sphinxstartmulticolumn{2}%
\begin{varwidth}[t]{\sphinxcolwidth{2}{2}}
\sphinxstyletheadfamily Splitter écrans (viewports)
\par
\vskip-\baselineskip\vbox{\hbox{\strut}}\end{varwidth}%
\sphinxstopmulticolumn
\\
\hline
\sphinxcode{\sphinxupquote{:sp}}
&
\begin{DUlineblock}{0em}
\item[] découper l’écran horizontalement
\end{DUlineblock}
\\
\hline
\sphinxcode{\sphinxupquote{:sp autrefichier}}
&
\begin{DUlineblock}{0em}
\item[] ouvrir autrefichier dans la seconde moitié de l’écran
\end{DUlineblock}
\\
\hline
\sphinxcode{\sphinxupquote{:vsp}}
&
\begin{DUlineblock}{0em}
\item[] découper l’écran verticalement
\end{DUlineblock}
\\
\hline
\sphinxcode{\sphinxupquote{Ctrl + w}} puis \sphinxcode{\sphinxupquote{Ctrl + {}`{}`w}}
&
\begin{DUlineblock}{0em}
\item[] navigue de viewport en viewport. Répétez l’opération plusieurs fois pour accéder au viewport désiré.
\end{DUlineblock}
\\
\hline
\sphinxcode{\sphinxupquote{Ctrl + w}} puis \sphinxcode{\sphinxupquote{j}}
&
\begin{DUlineblock}{0em}
\item[] déplace le curseur pour aller au viewport juste en dessous. La même chose fonctionne avec les touches \sphinxcode{\sphinxupquote{h}}, \sphinxcode{\sphinxupquote{k}}, \sphinxcode{\sphinxupquote{j}} et \sphinxcode{\sphinxupquote{l}} que l’on utilise traditionnellement pour se déplacer dans Vim.
\end{DUlineblock}
\\
\hline
\sphinxcode{\sphinxupquote{Ctrl + w}} puis \sphinxcode{\sphinxupquote{+}}
&
\begin{DUlineblock}{0em}
\item[] agrandit le viewport actuel.
\end{DUlineblock}
\\
\hline
\sphinxcode{\sphinxupquote{Ctrl + w}} puis \sphinxcode{\sphinxupquote{-}}
&
\begin{DUlineblock}{0em}
\item[] réduit le viewport actuel.
\end{DUlineblock}
\\
\hline
\sphinxcode{\sphinxupquote{Ctrl + w}} puis \sphinxcode{\sphinxupquote{=}}
&
\begin{DUlineblock}{0em}
\item[] égalise à nouveau la taille des viewports.
\end{DUlineblock}
\\
\hline
\sphinxcode{\sphinxupquote{Ctrl + w}} puis \sphinxcode{\sphinxupquote{r}}
&
\begin{DUlineblock}{0em}
\item[] échange la position des viewports. Fonctionne aussi avec « R » majuscule pour échanger en sens inverse.
\end{DUlineblock}
\\
\hline
\sphinxcode{\sphinxupquote{Ctrl + w}} puis \sphinxcode{\sphinxupquote{q}}
&
\begin{DUlineblock}{0em}
\item[] ferme le viewport actuel.
\end{DUlineblock}
\\
\hline
\end{tabular}
\par
\sphinxattableend\end{savenotes}

\begin{DUlineblock}{0em}
\item[] 
\end{DUlineblock}


\begin{savenotes}\sphinxattablestart
\centering
\begin{tabular}[t]{|*{2}{\X{1}{2}|}}
\hline
\sphinxstartmulticolumn{2}%
\begin{varwidth}[t]{\sphinxcolwidth{2}{2}}
\sphinxstyletheadfamily Zoom
\par
\vskip-\baselineskip\vbox{\hbox{\strut}}\end{varwidth}%
\sphinxstopmulticolumn
\\
\hline
\sphinxcode{\sphinxupquote{Ctrl + Shift}} + \sphinxcode{\sphinxupquote{+}}
&
\begin{DUlineblock}{0em}
\item[] zoom
\end{DUlineblock}
\\
\hline
\sphinxcode{\sphinxupquote{Ctrl + Shift}} + \sphinxcode{\sphinxupquote{-}}
&
\begin{DUlineblock}{0em}
\item[] dezoom
\end{DUlineblock}
\\
\hline
\end{tabular}
\par
\sphinxattableend\end{savenotes}


\section{Options de vim}
\label{\detokenize{20-vim:options-de-vim}}
Rem : pour qu’elles soient retenues, créer un fichier .vimrc dans le répertoire personnel (un exemple de fichier se trouve dans /etc/vim -\textgreater{} vimrc)


\begin{savenotes}\sphinxattablestart
\centering
\begin{tabular}[t]{|*{2}{\X{1}{2}|}}
\hline

\sphinxcode{\sphinxupquote{:set option}}
&
\begin{DUlineblock}{0em}
\item[] activer l’option en mode commande
\end{DUlineblock}
\\
\hline
\sphinxcode{\sphinxupquote{:set nooption}}
&
\begin{DUlineblock}{0em}
\item[] désactiver l’option en mode commande
\end{DUlineblock}
\\
\hline
\sphinxcode{\sphinxupquote{:set option?}}
&
\begin{DUlineblock}{0em}
\item[] connaitre l’état d’une option
\end{DUlineblock}
\\
\hline
\sphinxcode{\sphinxupquote{:set option=valeur}}
&
\begin{DUlineblock}{0em}
\item[] donner une valeur à une option
\end{DUlineblock}
\\
\hline
\sphinxcode{\sphinxupquote{:set syntax=ON}}
&
\begin{DUlineblock}{0em}
\item[] coloration synthaxique
\end{DUlineblock}
\\
\hline
\sphinxcode{\sphinxupquote{:set background=dark}}
&
\begin{DUlineblock}{0em}
\item[] coloration adaptée pour les fonds noirs
\end{DUlineblock}
\\
\hline
\sphinxcode{\sphinxupquote{:set number}}
&
\begin{DUlineblock}{0em}
\item[] affiche les n° de lignes
\end{DUlineblock}
\\
\hline
\sphinxcode{\sphinxupquote{:set showcmd}}
&
\begin{DUlineblock}{0em}
\item[] afficher la commande en cours
\end{DUlineblock}
\\
\hline
\sphinxcode{\sphinxupquote{:set ignorecase}}
&
\begin{DUlineblock}{0em}
\item[] ignorer la casse lors des recherches
\end{DUlineblock}
\\
\hline
\sphinxcode{\sphinxupquote{:set mouse=a}}
&
\begin{DUlineblock}{0em}
\item[] activer la souris
\end{DUlineblock}
\\
\hline
\end{tabular}
\par
\sphinxattableend\end{savenotes}


\section{Gérer les plugins}
\label{\detokenize{20-vim:gerer-les-plugins}}
\sphinxurl{https://artisan.karma-lab.net/configurer-vim}
\begin{itemize}
\item {} 
Création d’une arborescence pour les fichiers de config :

\end{itemize}

\fvset{hllines={, ,}}%
\begin{sphinxVerbatim}[commandchars=\\\{\}]
\PYGZdl{} \PYG{n+nb}{cd}
\PYGZdl{} mkdir \PYGZhy{}p .vim/\PYG{o}{\PYGZob{}}autoload,colors,syntax,plugin,spell,config\PYG{o}{\PYGZcb{}}
\PYGZdl{} mv .vimrc .vim/vimrc
\PYGZdl{} ln \PYGZhy{}s .vim/vimrc .vimrc
\end{sphinxVerbatim}

(on crée un lien vers \sphinxcode{\sphinxupquote{.vim/vimrc{}`}})
\begin{itemize}
\item {} 
Installation de \sphinxcode{\sphinxupquote{pathogen}} :

\end{itemize}

\fvset{hllines={, ,}}%
\begin{sphinxVerbatim}[commandchars=\\\{\}]
\PYGZdl{} \PYG{n+nb}{cd} \PYGZti{}/.vim
\PYGZdl{} git clone https://github.com/tpope/vim\PYGZhy{}pathogen.git pathogen
\PYGZdl{} \PYG{n+nb}{cd} autoload
\PYGZdl{} ln \PYGZhy{}s ../pathogen/autoload/pathogen.vim
\end{sphinxVerbatim}
\begin{itemize}
\item {} 
Pour mettre à jour \sphinxcode{\sphinxupquote{pathogen}} :

\end{itemize}

\fvset{hllines={, ,}}%
\begin{sphinxVerbatim}[commandchars=\\\{\}]
\PYGZdl{} \PYG{n+nb}{cd} \PYGZti{}/.vim/pathogen
\PYGZdl{} git pull
\end{sphinxVerbatim}
\begin{itemize}
\item {} 
Installer un plugin : exemple avec \sphinxcode{\sphinxupquote{NERDTree}}

\end{itemize}

\fvset{hllines={, ,}}%
\begin{sphinxVerbatim}[commandchars=\\\{\}]
\PYGZdl{} \PYG{n+nb}{cd} \PYGZti{}/.vim
\PYGZdl{} mkdir \PYGZhy{}p bundle
\PYGZdl{} \PYG{n+nb}{cd} bundle
\PYGZdl{} git clone https://github.com/scrooloose/nerdtree.git nerdtree
\end{sphinxVerbatim}
\begin{itemize}
\item {} 
Allure finale du dossier \sphinxcode{\sphinxupquote{.vim}} (situé dans \sphinxcode{\sphinxupquote{\textasciitilde{}/}})

\end{itemize}

\fvset{hllines={, ,}}%
\begin{sphinxVerbatim}[commandchars=\\\{\}]
.vim
├── autoload
│   └── pathogen.vim \PYGZhy{}\PYGZgt{} ../pathogen/autoload/pathogen.vim
├── bundle
│   └── nerdtree
│       ├──.......
├── colors
├── config
├── pathogen
│   ├── autoload
│   │   └── pathogen.vim
│   ├── CONTRIBUTING.markdown
│   └── README.markdown
├── plugin
├── spell
├── syntax
└── vimrc \PYG{o}{(}fichier\PYG{o}{)}
\end{sphinxVerbatim}
\begin{itemize}
\item {} 
Allure du fichier \sphinxcode{\sphinxupquote{vimrc}} :

\end{itemize}

\fvset{hllines={, ,}}%
\begin{sphinxVerbatim}[commandchars=\\\{\}]
1 set nocompatible
2
3 runtime! config/**/*.vim
4
5 set number
6
7 \PYGZdq{} Initialisation de pathogen
8 call pathogen\PYGZsh{}infect()
9 call pathogen\PYGZsh{}helptags()
10
\end{sphinxVerbatim}
\begin{itemize}
\item {} 
Pour démarrer NerdTree : taper \sphinxcode{\sphinxupquote{:NERDTree}} en mode interactif

\end{itemize}


\chapter{Scripts Shell}
\label{\detokenize{21-scripts-shell:scripts-shell}}\label{\detokenize{21-scripts-shell::doc}}
\begin{sphinxShadowBox}
\begin{itemize}
\item {} 
\phantomsection\label{\detokenize{21-scripts-shell:id1}}{\hyperref[\detokenize{21-scripts-shell:remarques-generales}]{\sphinxcrossref{Remarques générales}}}

\item {} 
\phantomsection\label{\detokenize{21-scripts-shell:id2}}{\hyperref[\detokenize{21-scripts-shell:variables}]{\sphinxcrossref{Variables}}}

\item {} 
\phantomsection\label{\detokenize{21-scripts-shell:id3}}{\hyperref[\detokenize{21-scripts-shell:quotes}]{\sphinxcrossref{Quotes}}}

\item {} 
\phantomsection\label{\detokenize{21-scripts-shell:id4}}{\hyperref[\detokenize{21-scripts-shell:read-demander-une-saisie}]{\sphinxcrossref{\sphinxcode{\sphinxupquote{read}} : Demander une saisie}}}

\item {} 
\phantomsection\label{\detokenize{21-scripts-shell:id5}}{\hyperref[\detokenize{21-scripts-shell:operations-mathematiques}]{\sphinxcrossref{Opérations mathématiques}}}

\item {} 
\phantomsection\label{\detokenize{21-scripts-shell:id6}}{\hyperref[\detokenize{21-scripts-shell:les-variables-d-environnement}]{\sphinxcrossref{Les variables d’environnement}}}

\item {} 
\phantomsection\label{\detokenize{21-scripts-shell:id7}}{\hyperref[\detokenize{21-scripts-shell:les-parametres}]{\sphinxcrossref{Les paramètres}}}

\item {} 
\phantomsection\label{\detokenize{21-scripts-shell:id8}}{\hyperref[\detokenize{21-scripts-shell:les-tableaux}]{\sphinxcrossref{Les Tableaux}}}

\item {} 
\phantomsection\label{\detokenize{21-scripts-shell:id9}}{\hyperref[\detokenize{21-scripts-shell:les-conditions}]{\sphinxcrossref{Les conditions}}}
\begin{itemize}
\item {} 
\phantomsection\label{\detokenize{21-scripts-shell:id10}}{\hyperref[\detokenize{21-scripts-shell:tests-sur-les-chaines-de-caracteres}]{\sphinxcrossref{Tests sur les chaines de caractères}}}

\item {} 
\phantomsection\label{\detokenize{21-scripts-shell:id11}}{\hyperref[\detokenize{21-scripts-shell:tests-sur-les-nombres}]{\sphinxcrossref{Tests sur les nombres}}}

\item {} 
\phantomsection\label{\detokenize{21-scripts-shell:id12}}{\hyperref[\detokenize{21-scripts-shell:tests-sur-les-fichiers}]{\sphinxcrossref{Tests sur les fichiers}}}

\item {} 
\phantomsection\label{\detokenize{21-scripts-shell:id13}}{\hyperref[\detokenize{21-scripts-shell:plusieurs-tests-et}]{\sphinxcrossref{Plusieurs tests : \sphinxcode{\sphinxupquote{\&\&}} et \sphinxcode{\sphinxupquote{\textbar{}\textbar{}}}}}}

\item {} 
\phantomsection\label{\detokenize{21-scripts-shell:id14}}{\hyperref[\detokenize{21-scripts-shell:test-inverse-not}]{\sphinxcrossref{Test inversé : \sphinxcode{\sphinxupquote{not}}}}}

\end{itemize}

\item {} 
\phantomsection\label{\detokenize{21-scripts-shell:id15}}{\hyperref[\detokenize{21-scripts-shell:switch-case}]{\sphinxcrossref{Switch case}}}

\item {} 
\phantomsection\label{\detokenize{21-scripts-shell:id16}}{\hyperref[\detokenize{21-scripts-shell:les-boucles}]{\sphinxcrossref{Les boucles}}}
\begin{itemize}
\item {} 
\phantomsection\label{\detokenize{21-scripts-shell:id17}}{\hyperref[\detokenize{21-scripts-shell:boucle-while}]{\sphinxcrossref{Boucle \sphinxcode{\sphinxupquote{while}}}}}

\item {} 
\phantomsection\label{\detokenize{21-scripts-shell:id18}}{\hyperref[\detokenize{21-scripts-shell:boucle-until}]{\sphinxcrossref{Boucle \sphinxcode{\sphinxupquote{until}}}}}

\item {} 
\phantomsection\label{\detokenize{21-scripts-shell:id19}}{\hyperref[\detokenize{21-scripts-shell:boucle-for}]{\sphinxcrossref{Boucle \sphinxcode{\sphinxupquote{for}}}}}

\end{itemize}

\item {} 
\phantomsection\label{\detokenize{21-scripts-shell:id20}}{\hyperref[\detokenize{21-scripts-shell:les-fonctions}]{\sphinxcrossref{Les fonctions}}}

\end{itemize}
\end{sphinxShadowBox}


\section{Remarques générales}
\label{\detokenize{21-scripts-shell:remarques-generales}}\begin{itemize}
\item {} 
ici on utilisera bash

\item {} 
Extension : \sphinxcode{\sphinxupquote{fichier.sh}}

\item {} 
ajouter \sphinxcode{\sphinxupquote{\#!/bin/bash}} au début pour s’assurer qu’il est éxécuté avec bash et pas un autre shell

\item {} 
comentaires : \sphinxcode{\sphinxupquote{\#}}

\item {} \begin{description}
\item[{donner le droit « éxécutable » au script :}] \leavevmode
\sphinxcode{\sphinxupquote{chmod +x essai.sh}}

\end{description}

\item {} \begin{description}
\item[{Exécuter le script :}] \leavevmode
\sphinxcode{\sphinxupquote{./essai.sh}}

\end{description}

\item {} \begin{description}
\item[{Mode debug}] \leavevmode{[}\sphinxcode{\sphinxupquote{bash -x}}{]}
\sphinxcode{\sphinxupquote{bash -x essai.sh}}

\end{description}

\item {} \begin{description}
\item[{Pour éxécuter un script à n’importe quel endroit -\textgreater{} copier le script dans un des répertoires du \sphinxcode{\sphinxupquote{PATH}}}] \leavevmode
\sphinxcode{\sphinxupquote{echo \$PATH}} pour voir où se trouvent les répertoires du PATH

\end{description}

\end{itemize}


\section{Variables}
\label{\detokenize{21-scripts-shell:variables}}\begin{quote}

\sphinxstylestrong{Attention pas d’espaces autour des {}`{}`={}`{}`}
\end{quote}
\begin{description}
\item[{\sphinxcode{\sphinxupquote{echo something}}}] \leavevmode
renvoie le paramètre something

\item[{\sphinxcode{\sphinxupquote{echo \$something}}}] \leavevmode
pour afficher une variable dans un script (ajouter “\$”)

\item[{\sphinxcode{\sphinxupquote{echo -e "test\textbackslash{}ntes2\textbackslash{}n"}}}] \leavevmode
tient comptre des retours à la ligne \sphinxcode{\sphinxupquote{\textbackslash{}n}}

\end{description}


\section{Quotes}
\label{\detokenize{21-scripts-shell:quotes}}\begin{itemize}
\item {} 
\sphinxstylestrong{Simple quotes} : bash ne tient pas compte de la variable quand on utilise des simple quotes \sphinxcode{\sphinxupquote{'...'}}

\end{itemize}

\fvset{hllines={, ,}}%
\begin{sphinxVerbatim}[commandchars=\\\{\}]
\PYGZdl{} \PYG{n+nv}{message}\PYG{o}{=}\PYG{l+s+s1}{\PYGZsq{}Bonjour tout le monde\PYGZsq{}}
\PYGZdl{} \PYG{n+nb}{echo} \PYG{l+s+s1}{\PYGZsq{}Le message est : \PYGZdl{}message\PYGZsq{}}
Le message est : \PYG{n+nv}{\PYGZdl{}message}\PYG{l+s+sb}{{}`}\PYG{l+s+sb}{{}`}
\end{sphinxVerbatim}
\begin{itemize}
\item {} 
\sphinxstylestrong{Double quotes} : bash tient compte de la variable quand on utilise des double quotes \sphinxcode{\sphinxupquote{"..."}}

\end{itemize}

\fvset{hllines={, ,}}%
\begin{sphinxVerbatim}[commandchars=\\\{\}]
\PYGZdl{} \PYG{n+nv}{message}\PYG{o}{=}\PYG{l+s+s1}{\PYGZsq{}Bonjour tout le monde\PYGZsq{}}
\PYGZdl{} \PYG{n+nb}{echo} \PYG{l+s+s2}{\PYGZdq{}Le message est : Bonjour tout le monde\PYGZdq{}}
Le message est : \PYG{n+nv}{\PYGZdl{}message}
\end{sphinxVerbatim}
\begin{itemize}
\item {} 
\sphinxstylestrong{Back quotes} : bash éxécute ce qui se trouve dans les back quotes \sphinxcode{\sphinxupquote{{}`...{}`}}

\end{itemize}

\fvset{hllines={, ,}}%
\begin{sphinxVerbatim}[commandchars=\\\{\}]
\PYGZdl{} \PYG{n+nv}{message}\PYG{o}{=}\PYG{l+s+sb}{{}`}\PYG{n+nb}{pwd}\PYG{l+s+sb}{{}`}
\PYGZdl{} \PYG{n+nb}{echo} \PYG{l+s+s2}{\PYGZdq{}}\PYG{l+s+s2}{Vous êtes dans le dossier }\PYG{n+nv}{\PYGZdl{}message}\PYG{l+s+s2}{\PYGZdq{}}
Vous êtes dans le dossier /home/mateo21/bin
\end{sphinxVerbatim}


\section{\sphinxstyleliteralintitle{\sphinxupquote{read}} : Demander une saisie}
\label{\detokenize{21-scripts-shell:read-demander-une-saisie}}\begin{itemize}
\item {} 
\sphinxcode{\sphinxupquote{-p}} : demande un message de prompt

\end{itemize}

\fvset{hllines={, ,}}%
\begin{sphinxVerbatim}[commandchars=\\\{\}]
\PYGZdl{} \PYG{n+nb}{read} \PYGZhy{}p \PYG{l+s+s1}{\PYGZsq{}Entrez votre nom : \PYGZsq{}} nom prenom
\PYGZdl{} \PYG{n+nb}{echo} \PYG{l+s+s2}{\PYGZdq{}}\PYG{l+s+s2}{Bonjour }\PYG{n+nv}{\PYGZdl{}nom}\PYG{l+s+s2}{ }\PYG{n+nv}{\PYGZdl{}prenom}\PYG{l+s+s2}{ !}\PYG{l+s+s2}{\PYGZdq{}}
Entrez votre nom : Mathieu
Bonjour Deschamps Mathieu !
\end{sphinxVerbatim}
\begin{itemize}
\item {} 
\sphinxcode{\sphinxupquote{-n}} : nb max de caractères

\end{itemize}

\fvset{hllines={, ,}}%
\begin{sphinxVerbatim}[commandchars=\\\{\}]
\PYGZdl{} \PYG{n+nb}{read} \PYGZhy{}p \PYG{l+s+s1}{\PYGZsq{}Entrez votre login (5 caractères max) : \PYGZsq{}} \PYGZhy{}n \PYG{l+m}{5} nom
\PYGZdl{} \PYG{n+nb}{echo} \PYG{l+s+s2}{\PYGZdq{}}\PYG{l+s+s2}{Bonjour }\PYG{n+nv}{\PYGZdl{}nom}\PYG{l+s+s2}{ !}\PYG{l+s+s2}{\PYGZdq{}}
\end{sphinxVerbatim}
\begin{itemize}
\item {} 
\sphinxcode{\sphinxupquote{-t}} : précise un temps max

\end{itemize}

\fvset{hllines={, ,}}%
\begin{sphinxVerbatim}[commandchars=\\\{\}]
\PYGZdl{} \PYG{n+nb}{read} \PYGZhy{}p \PYG{l+s+s1}{\PYGZsq{}Entrez le code de désamorçage de la bombe (vous avez 5 secondes) : \PYGZsq{}}  \PYGZhy{}t \PYG{l+m}{5} code
\PYGZdl{} \PYG{n+nb}{echo} \PYGZhy{}e \PYG{l+s+s2}{\PYGZdq{}\PYGZbs{}nBoum !\PYGZdq{}}
\end{sphinxVerbatim}
\begin{itemize}
\item {} 
\sphinxcode{\sphinxupquote{-s}} : ne pas afficher les caractères saisis

\end{itemize}

\fvset{hllines={, ,}}%
\begin{sphinxVerbatim}[commandchars=\\\{\}]
\PYGZdl{} \PYG{n+nb}{read} \PYGZhy{}p \PYG{l+s+s1}{\PYGZsq{}Entrez votre mot de passe : \PYGZsq{}} \PYGZhy{}s pass
\PYGZdl{} \PYG{n+nb}{echo} \PYGZhy{}e \PYG{l+s+s2}{\PYGZdq{}}\PYG{l+s+s2}{\PYGZbs{}nMerci ! Je vais dire à tout le monde que votre mot de passe est }\PYG{n+nv}{\PYGZdl{}pass}\PYG{l+s+s2}{ ! :)}\PYG{l+s+s2}{\PYGZdq{}}
\end{sphinxVerbatim}


\section{Opérations mathématiques}
\label{\detokenize{21-scripts-shell:operations-mathematiques}}
\fvset{hllines={, ,}}%
\begin{sphinxVerbatim}[commandchars=\\\{\}]
\PYGZdl{} \PYG{n+nb}{let} \PYG{l+s+s2}{\PYGZdq{}a = 5\PYGZdq{}}
\PYGZdl{} \PYG{n+nb}{let} \PYG{l+s+s2}{\PYGZdq{}b = 2\PYGZdq{}}
\PYGZdl{} \PYG{n+nb}{let} \PYG{l+s+s2}{\PYGZdq{}c = a + b\PYGZdq{}}
\PYG{l+m}{7}
\end{sphinxVerbatim}

Remarque : on peut utiliser : \sphinxcode{\sphinxupquote{let "a += 5"}}

\sphinxstylestrong{Opérations utilisables :}
\begin{itemize}
\item {} 
l’addition : \sphinxcode{\sphinxupquote{+}}

\item {} 
la soustraction : \sphinxcode{\sphinxupquote{-}}

\item {} 
la multiplication : \sphinxcode{\sphinxupquote{*}}

\item {} 
la division : \sphinxcode{\sphinxupquote{/}}

\item {} 
la puissance : \sphinxcode{\sphinxupquote{**}}

\item {} 
le modulo (renvoie le reste de la division entière) : \sphinxcode{\sphinxupquote{\%}}

\end{itemize}

\sphinxstylestrong{Opérations sur des nombres décimaux} : cf commande \sphinxcode{\sphinxupquote{bc}}


\section{Les variables d’environnement}
\label{\detokenize{21-scripts-shell:les-variables-d-environnement}}
Elles sont disponibles tout le temps pour tous les scripts (variables globales) On les met en majuscules par convention.

Exemples :
\begin{itemize}
\item {} 
\sphinxcode{\sphinxupquote{SHELL}} : indique quel type de shell est en cours d’utilisation (sh, bash, ksh…) ;

\item {} 
\sphinxcode{\sphinxupquote{PATH}} : une liste des répertoires qui contiennent des exécutables que vous souhaitez pouvoir lancer sans indiquer leur répertoire. Nous en avons parlé un peu plus tôt. Si un programme se trouve dans un de ces dossiers, vous pourrez l’invoquer quel que soit le dossier dans lequel vous vous trouvez ;

\item {} 
\sphinxcode{\sphinxupquote{EDITOR}} : l’éditeur de texte par défaut qui s’ouvre lorsque cela est nécessaire ;

\item {} 
\sphinxcode{\sphinxupquote{HOME}} : la position de votre dossier home ;

\item {} 
\sphinxcode{\sphinxupquote{PWD}} : le dossier dans lequel vous vous trouvez ;

\item {} 
\sphinxcode{\sphinxupquote{OLDPWD}} : le dossier dans lequel vous vous trouviez auparavant.

\end{itemize}

Pour en créer de nouvelles : \sphinxcode{\sphinxupquote{export}}


\section{Les paramètres}
\label{\detokenize{21-scripts-shell:les-parametres}}
\sphinxcode{\sphinxupquote{./script.sh param1 param2 param3}}
\begin{description}
\item[{Pour récupérer ces paramètres dans le script :}] \leavevmode\begin{itemize}
\item {} 
\sphinxcode{\sphinxupquote{\$\#}} : contient le nombre de paramètres

\item {} 
\sphinxcode{\sphinxupquote{\$0}} : contient le nom du script exécuté (ici ./variables.sh)

\item {} 
\sphinxcode{\sphinxupquote{\$1}} : contient le premier paramètre

\item {} 
\sphinxcode{\sphinxupquote{\$2}} : contient le second paramètre

\item {} 
…

\item {} 
\sphinxcode{\sphinxupquote{\$9}} : contient le 9e paramètre

\end{itemize}

\end{description}

Si on a plus de 9 variables : décaler l’ordre avec la commande \sphinxcode{\sphinxupquote{shift}} . Exemple :
\begin{quote}

Script :

\fvset{hllines={, ,}}%
\begin{sphinxVerbatim}[commandchars=\\\{\}]
\PYG{n+nb}{echo} \PYG{l+s+s2}{\PYGZdq{}}\PYG{l+s+s2}{Le paramètre 1 est }\PYG{n+nv}{\PYGZdl{}1}\PYG{l+s+s2}{\PYGZdq{}}
\PYG{n+nb}{shift}
\PYG{n+nb}{echo} \PYG{l+s+s2}{\PYGZdq{}}\PYG{l+s+s2}{Le paramètre 1 est maintenant }\PYG{n+nv}{\PYGZdl{}1}\PYG{l+s+s2}{\PYGZdq{}}
\end{sphinxVerbatim}

A l’exécution

\fvset{hllines={, ,}}%
\begin{sphinxVerbatim}[commandchars=\\\{\}]
\PYGZdl{} ./variables.sh param1 param2 param3
Le paramètre \PYG{l+m}{1} est param1
Le paramètre \PYG{l+m}{1} est maintenant param2
\end{sphinxVerbatim}
\end{quote}


\section{Les Tableaux}
\label{\detokenize{21-scripts-shell:les-tableaux}}\begin{quote}
\begin{itemize}
\item {} \begin{description}
\item[{Définir un tableau :}] \leavevmode
\sphinxcode{\sphinxupquote{tableau=('valeur0' 'valeur1' 'valeur2')}}

\end{description}

\item {} \begin{description}
\item[{Accéder à une case particulière :}] \leavevmode
\sphinxcode{\sphinxupquote{\$\{tableau{[}2{]}\}}}

\end{description}

\item {} \begin{description}
\item[{Définir le contenu d’une case :}] \leavevmode
\sphinxcode{\sphinxupquote{tableau{[}2{]}='valeur2'}}

\end{description}

\item {} 
On peut sauter des cases :

\end{itemize}

\fvset{hllines={, ,}}%
\begin{sphinxVerbatim}[commandchars=\\\{\}]
\PYG{n+nv}{tableau}\PYG{o}{=}\PYG{o}{(}\PYG{l+s+s1}{\PYGZsq{}valeur0\PYGZsq{}} \PYG{l+s+s1}{\PYGZsq{}valeur1\PYGZsq{}} \PYG{l+s+s1}{\PYGZsq{}valeur2\PYGZsq{}}\PYG{o}{)}
tableau\PYG{o}{[}\PYG{l+m}{5}\PYG{o}{]}\PYG{o}{=}\PYG{l+s+s1}{\PYGZsq{}valeur5\PYGZsq{}}
\PYG{n+nb}{echo} \PYG{l+s+si}{\PYGZdl{}\PYGZob{}}\PYG{n+nv}{tableau}\PYG{p}{[1]}\PYG{l+s+si}{\PYGZcb{}}
\end{sphinxVerbatim}
\begin{itemize}
\item {} \begin{description}
\item[{Afficher l’ensemble d’une tableau :}] \leavevmode
\sphinxcode{\sphinxupquote{\$\{tableau{[}*{]}\}}}

\end{description}

\end{itemize}
\end{quote}


\section{Les conditions}
\label{\detokenize{21-scripts-shell:les-conditions}}
Synthaxe :

\fvset{hllines={, ,}}%
\begin{sphinxVerbatim}[commandchars=\\\{\}]
\PYG{k}{if} \PYG{o}{[} \PYG{n+nv}{\PYGZdl{}1} \PYG{o}{=} \PYG{l+s+s2}{\PYGZdq{}Bruno\PYGZdq{}} \PYG{o}{]}     \PYG{c+c1}{\PYGZsh{} ou if [ \PYGZdl{}1 = Bruno\PYGZdq{} ]; then}
\PYG{k}{then}
        \PYG{n+nb}{echo} \PYG{l+s+s2}{\PYGZdq{}Salut Bruno !\PYGZdq{}}
\PYG{k}{elif} \PYG{o}{[} \PYG{n+nv}{\PYGZdl{}1} \PYG{o}{=} \PYG{l+s+s2}{\PYGZdq{}Michel\PYGZdq{}} \PYG{o}{]}
\PYG{k}{then}
        \PYG{n+nb}{echo} \PYG{l+s+s2}{\PYGZdq{}Bien le bonjour Michel\PYGZdq{}}
\PYG{k}{elif} \PYG{o}{[} \PYG{n+nv}{\PYGZdl{}1} \PYG{o}{=} \PYG{l+s+s2}{\PYGZdq{}Jean\PYGZdq{}} \PYG{o}{]}
\PYG{k}{then}
        \PYG{n+nb}{echo} \PYG{l+s+s2}{\PYGZdq{}Hé Jean, ça va ?\PYGZdq{}}
\PYG{k}{else}
        \PYG{n+nb}{echo} \PYG{l+s+s2}{\PYGZdq{}J\PYGZsq{}te connais pas, ouste !\PYGZdq{}}
\PYG{k}{fi}
\end{sphinxVerbatim}


\subsection{Tests sur les chaines de caractères}
\label{\detokenize{21-scripts-shell:tests-sur-les-chaines-de-caracteres}}
Rappel : toutes les variables sont traitées comme des strings
\begin{description}
\item[{\sphinxcode{\sphinxupquote{\$chaine1 = \$chaine2}}}] \leavevmode
Vérifie si les deux chaînes sont identiques. Notez que bash est sensible à la casse : « b » est donc différent de « B ».

Il est aussi possible d’écrire \sphinxcode{\sphinxupquote{==}} pour les habitués du langage C.

\item[{\sphinxcode{\sphinxupquote{\$chaine1 != \$chaine2}}}] \leavevmode
Vérifie si les deux chaînes sont différentes.

\item[{\sphinxcode{\sphinxupquote{-z \$chaine}}}] \leavevmode
Vérifie si la chaîne est vide.

\item[{\sphinxcode{\sphinxupquote{-n \$chaine}}}] \leavevmode
Vérifie si la chaîne est non vide.

\end{description}


\subsection{Tests sur les nombres}
\label{\detokenize{21-scripts-shell:tests-sur-les-nombres}}\begin{description}
\item[{\sphinxcode{\sphinxupquote{\$num1 -eq \$num2}}}] \leavevmode
Vérifie si les nombres sont égaux (equal). À ne pas confondre avec le « = » qui, lui, compare deux chaînes de caractères.

\item[{\sphinxcode{\sphinxupquote{\$num1 -ne \$num2}}}] \leavevmode
Vérifie si les nombres sont différents (nonequal).

\end{description}

\sphinxstylestrong{Encore une fois, ne confondez pas avec {}`{}`!={}`{}` qui est censé être utilisé sur des chaînes de caractères.}
\begin{description}
\item[{\sphinxcode{\sphinxupquote{\$num1 -lt \$num2}}}] \leavevmode
Vérifie si num1 est inférieur ( \textless{} ) à num2 (lowerthan).

\item[{\sphinxcode{\sphinxupquote{\$num1 -le \$num2}}}] \leavevmode
Vérifie si num1 est inférieur ou égal ( \textless{}= ) à num2 (lowerorequal).

\item[{\sphinxcode{\sphinxupquote{\$num1 -gt \$num2}}}] \leavevmode
Vérifie si num1 est supérieur ( \textgreater{} ) à num2 (greaterthan).

\item[{\sphinxcode{\sphinxupquote{\$num1 -ge \$num2}}}] \leavevmode
Vérifie si num1 est supérieur ou égal ( \textgreater{}= ) à num2 (greaterorequal).

\end{description}


\subsection{Tests sur les fichiers}
\label{\detokenize{21-scripts-shell:tests-sur-les-fichiers}}\begin{description}
\item[{\sphinxcode{\sphinxupquote{-e \$nomfichier}}}] \leavevmode
Vérifie si le fichier existe.

\item[{\sphinxcode{\sphinxupquote{-d \$nomfichier}}}] \leavevmode
Vérifie si le fichier est un répertoire. N’oubliez pas que sous Linux, tout est considéré comme un fichier, même un répertoire !

\item[{\sphinxcode{\sphinxupquote{-f \$nomfichier}}}] \leavevmode
Vérifie si le fichier est un… fichier. Un vrai fichier cette fois, pas un dossier.

\item[{\sphinxcode{\sphinxupquote{-L \$nomfichier}}}] \leavevmode
Vérifie si le fichier est un lien symbolique (raccourci).

\item[{\sphinxcode{\sphinxupquote{-r \$nomfichier}}}] \leavevmode
Vérifie si le fichier est lisible (r).

\item[{\sphinxcode{\sphinxupquote{-w \$nomfichier}}}] \leavevmode
Vérifie si le fichier est modifiable (w).

\item[{\sphinxcode{\sphinxupquote{-x \$nomfichier}}}] \leavevmode
Vérifie si le fichier est exécutable (x).

\item[{\sphinxcode{\sphinxupquote{\$fichier1 -nt \$fichier2}}}] \leavevmode
Vérifie si fichier1 est plus récent que fichier2 (newerthan).

\item[{\sphinxcode{\sphinxupquote{\$fichier1 -ot \$fichier2}}}] \leavevmode
Vérifie si fichier1 est plus vieux que fichier2 (olderthan).

\end{description}


\subsection{Plusieurs tests : \sphinxstyleliteralintitle{\sphinxupquote{\&\&}} et \sphinxstyleliteralintitle{\sphinxupquote{\textbar{}\textbar{}}}}
\label{\detokenize{21-scripts-shell:plusieurs-tests-et}}
Exemple :

\fvset{hllines={, ,}}%
\begin{sphinxVerbatim}[commandchars=\\\{\}]
\PYG{k}{if} \PYG{o}{[} \PYG{n+nv}{\PYGZdl{}\PYGZsh{}} \PYGZhy{}ge \PYG{l+m}{1} \PYG{o}{]} \PYG{o}{\PYGZam{}\PYGZam{}} \PYG{o}{[} \PYG{n+nv}{\PYGZdl{}1} \PYG{o}{=} \PYG{l+s+s1}{\PYGZsq{}koala\PYGZsq{}} \PYG{o}{]}
\PYG{k}{then}
        \PYG{n+nb}{echo} \PYG{l+s+s2}{\PYGZdq{}Bravo !\PYGZdq{}}
        \PYG{n+nb}{echo} \PYG{l+s+s2}{\PYGZdq{}Vous connaissez le mot de passe\PYGZdq{}}
\PYG{k}{else}
        \PYG{n+nb}{echo} \PYG{l+s+s2}{\PYGZdq{}Vous n\PYGZsq{}avez pas le bon mot de passe\PYGZdq{}}
\PYG{k}{fi}
\end{sphinxVerbatim}


\subsection{Test inversé : \sphinxstyleliteralintitle{\sphinxupquote{not}}}
\label{\detokenize{21-scripts-shell:test-inverse-not}}
Exemple :

\fvset{hllines={, ,}}%
\begin{sphinxVerbatim}[commandchars=\\\{\}]
\PYG{k}{if} \PYG{o}{[} ! \PYGZhy{}e fichier \PYG{o}{]}
\PYG{k}{then}
        \PYG{n+nb}{echo} \PYG{l+s+s2}{\PYGZdq{}Le fichier n\PYGZsq{}existe pas\PYGZdq{}}
\PYG{k}{fi}
\end{sphinxVerbatim}


\section{Switch case}
\label{\detokenize{21-scripts-shell:switch-case}}
Exemple :

\fvset{hllines={, ,}}%
\begin{sphinxVerbatim}[commandchars=\\\{\}]
\PYG{k}{case} \PYG{n+nv}{\PYGZdl{}1} in
        \PYG{l+s+s2}{\PYGZdq{}Bruno\PYGZdq{}}\PYG{o}{)}
                \PYG{n+nb}{echo} \PYG{l+s+s2}{\PYGZdq{}Salut Bruno !\PYGZdq{}}
                \PYG{n+nb}{echo} \PYG{l+s+s2}{\PYGZdq{}tu vas bien ?\PYGZdq{}}
                \PYG{p}{;}\PYG{p}{;}
        \PYG{l+s+s2}{\PYGZdq{}Michel\PYGZdq{}}\PYG{o}{)}
                \PYG{n+nb}{echo} \PYG{l+s+s2}{\PYGZdq{}Bien le bonjour Michel\PYGZdq{}}
                \PYG{p}{;}\PYG{p}{;}
        \PYG{l+s+s2}{\PYGZdq{}Jean\PYGZdq{}}\PYG{o}{)}
                \PYG{n+nb}{echo} \PYG{l+s+s2}{\PYGZdq{}Hé Jean, ça va ?\PYGZdq{}}
                \PYG{p}{;}\PYG{p}{;}
        *\PYG{o}{)}
                \PYG{n+nb}{echo} \PYG{l+s+s2}{\PYGZdq{}J\PYGZsq{}te connais pas, ouste !\PYGZdq{}}
                \PYG{p}{;}\PYG{p}{;}
\PYG{k}{esac}
\end{sphinxVerbatim}

Exemple 2 : avec des ou (attention : \sphinxcode{\sphinxupquote{\textbar{}}} et pas \sphinxcode{\sphinxupquote{\textbar{}\textbar{}}})

\fvset{hllines={, ,}}%
\begin{sphinxVerbatim}[commandchars=\\\{\}]
\PYG{k}{case} \PYG{n+nv}{\PYGZdl{}1} in
        \PYG{l+s+s2}{\PYGZdq{}Chien\PYGZdq{}} \PYG{p}{\textbar{}} \PYG{l+s+s2}{\PYGZdq{}Chat\PYGZdq{}} \PYG{p}{\textbar{}} \PYG{l+s+s2}{\PYGZdq{}Souris\PYGZdq{}}\PYG{o}{)}
                \PYG{n+nb}{echo} \PYG{l+s+s2}{\PYGZdq{}C\PYGZsq{}est un mammifère\PYGZdq{}}
                \PYG{p}{;}\PYG{p}{;}
        \PYG{l+s+s2}{\PYGZdq{}Moineau\PYGZdq{}} \PYG{p}{\textbar{}} \PYG{l+s+s2}{\PYGZdq{}Pigeon\PYGZdq{}}\PYG{o}{)}
                \PYG{n+nb}{echo} \PYG{l+s+s2}{\PYGZdq{}C\PYGZsq{}est un oiseau\PYGZdq{}}
                \PYG{p}{;}\PYG{p}{;}
        *\PYG{o}{)}
                \PYG{n+nb}{echo} \PYG{l+s+s2}{\PYGZdq{}Je ne sais pas ce que c\PYGZsq{}est\PYGZdq{}}
                \PYG{p}{;}\PYG{p}{;}
\PYG{k}{esac}
\end{sphinxVerbatim}


\section{Les boucles}
\label{\detokenize{21-scripts-shell:les-boucles}}

\subsection{Boucle \sphinxstyleliteralintitle{\sphinxupquote{while}}}
\label{\detokenize{21-scripts-shell:boucle-while}}
Exemple :

\fvset{hllines={, ,}}%
\begin{sphinxVerbatim}[commandchars=\\\{\}]
\PYG{c+c1}{\PYGZsh{} La réponse est vide ? Différente de \PYGZdq{}oui\PYGZdq{} ?}
\PYG{k}{while} \PYG{o}{[} \PYGZhy{}z \PYG{n+nv}{\PYGZdl{}reponse} \PYG{o}{]} \PYG{o}{\textbar{}\textbar{}} \PYG{o}{[} \PYG{n+nv}{\PYGZdl{}reponse} !\PYG{o}{=} \PYG{l+s+s1}{\PYGZsq{}oui\PYGZsq{}} \PYG{o}{]}
\PYG{k}{do}
        \PYG{n+nb}{read} \PYGZhy{}p \PYG{l+s+s1}{\PYGZsq{}Dites oui : \PYGZsq{}} reponse
\PYG{k}{done}
\end{sphinxVerbatim}


\subsection{Boucle \sphinxstyleliteralintitle{\sphinxupquote{until}}}
\label{\detokenize{21-scripts-shell:boucle-until}}
\sphinxcode{\sphinxupquote{while}} s’exécute tant que la condition est vraie, \sphinxcode{\sphinxupquote{until}} jusqu’à ce qu’elle le soit

Exemple :

\fvset{hllines={, ,}}%
\begin{sphinxVerbatim}[commandchars=\\\{\}]
\PYG{c+c1}{\PYGZsh{} Boucle jusqu\PYGZsq{}à ce que la réponse soit non vide et qu\PYGZsq{}elle soit \PYGZdq{}oui\PYGZdq{}}
\PYG{k}{until} \PYG{o}{[} \PYGZhy{}n \PYG{n+nv}{\PYGZdl{}reponse} \PYG{o}{]} \PYG{o}{\PYGZam{}\PYGZam{}} \PYG{o}{[} \PYG{n+nv}{\PYGZdl{}reponse} \PYG{o}{=}\PYG{o}{=} \PYG{l+s+s1}{\PYGZsq{}oui\PYGZsq{}} \PYG{o}{]}
\PYG{k}{do}
        \PYG{n+nb}{read} \PYGZhy{}p \PYG{l+s+s1}{\PYGZsq{}Dites oui : \PYGZsq{}} reponse
\PYG{k}{done}
\end{sphinxVerbatim}


\subsection{Boucle \sphinxstyleliteralintitle{\sphinxupquote{for}}}
\label{\detokenize{21-scripts-shell:boucle-for}}
Exemple :

\fvset{hllines={, ,}}%
\begin{sphinxVerbatim}[commandchars=\\\{\}]
\PYG{k}{for} variable in \PYG{l+s+s1}{\PYGZsq{}valeur1\PYGZsq{}} \PYG{l+s+s1}{\PYGZsq{}valeur2\PYGZsq{}} \PYG{l+s+s1}{\PYGZsq{}valeur3\PYGZsq{}}
 \PYG{k}{do}
         \PYG{n+nb}{echo} \PYG{l+s+s2}{\PYGZdq{}}\PYG{l+s+s2}{La variable vaut }\PYG{n+nv}{\PYGZdl{}variable}\PYG{l+s+s2}{\PYGZdq{}}
 \PYG{k}{done}
\end{sphinxVerbatim}

Exemple 2 :

\fvset{hllines={, ,}}%
\begin{sphinxVerbatim}[commandchars=\\\{\}]
\PYG{n+nv}{liste\PYGZus{}fichiers}\PYG{o}{=}\PYG{l+s+sb}{{}`}ls\PYG{l+s+sb}{{}`}

\PYG{k}{for} fichier in \PYG{n+nv}{\PYGZdl{}liste\PYGZus{}fichiers}
\PYG{k}{do}
        \PYG{n+nb}{echo} \PYG{l+s+s2}{\PYGZdq{}}\PYG{l+s+s2}{Fichier trouvé : }\PYG{n+nv}{\PYGZdl{}fichier}\PYG{l+s+s2}{\PYGZdq{}}
\PYG{k}{done}
\end{sphinxVerbatim}

Exemple 3 :

\fvset{hllines={, ,}}%
\begin{sphinxVerbatim}[commandchars=\\\{\}]
\PYG{n+nv}{liste\PYGZus{}fichiers}\PYG{o}{=}\PYG{l+s+sb}{{}`}ls\PYG{l+s+sb}{{}`}

\PYG{k}{for} fichier in \PYG{n+nv}{\PYGZdl{}liste\PYGZus{}fichiers}
\PYG{k}{do}
        \PYG{n+nb}{echo} \PYG{l+s+s2}{\PYGZdq{}}\PYG{l+s+s2}{Fichier trouvé : }\PYG{n+nv}{\PYGZdl{}fichier}\PYG{l+s+s2}{\PYGZdq{}}
\PYG{k}{done}
\end{sphinxVerbatim}

Exemple 4 : renommer chacun des fichiers du répertoire actuel en leur ajoutant un suffixe -old par exemple

\fvset{hllines={, ,}}%
\begin{sphinxVerbatim}[commandchars=\\\{\}]
\PYG{k}{for} fichier in \PYG{l+s+sb}{{}`}ls\PYG{l+s+sb}{{}`}
\PYG{k}{do}
        mv \PYG{n+nv}{\PYGZdl{}fichier} \PYG{n+nv}{\PYGZdl{}fichier}\PYGZhy{}old
\PYG{k}{done}
\end{sphinxVerbatim}

Exemple 5 : pour simuler une boucle for plus classique (seq génère tous les nombres allant du premier paramètre au dernier paramètre)

\fvset{hllines={, ,}}%
\begin{sphinxVerbatim}[commandchars=\\\{\}]
\PYG{k}{for} i in \PYG{l+s+sb}{{}`}seq \PYG{l+m}{1} \PYG{l+m}{10}\PYG{l+s+sb}{{}`}\PYG{p}{;}
\PYG{k}{do}
        \PYG{n+nb}{echo} \PYG{n+nv}{\PYGZdl{}i}
\PYG{k}{done}
\end{sphinxVerbatim}

Remarque : pour aller de 2 en 2 : \sphinxcode{\sphinxupquote{seq 1 2 10}}


\section{Les fonctions}
\label{\detokenize{21-scripts-shell:les-fonctions}}
Déclarer une fonction : 2 possibilités

\fvset{hllines={, ,}}%
\begin{sphinxVerbatim}[commandchars=\\\{\}]
maFonction \PYG{o}{(}\PYG{o}{)}
\PYG{o}{\PYGZob{}}
        bloc d’instructions
\PYG{o}{\PYGZcb{}}
\end{sphinxVerbatim}

ou :

\fvset{hllines={, ,}}%
\begin{sphinxVerbatim}[commandchars=\\\{\}]
\PYG{k}{function} maFonction
\PYG{o}{\PYGZob{}}
        bloc d’instructions
\PYG{o}{\PYGZcb{}}
\end{sphinxVerbatim}

Appeler une fonction :

\fvset{hllines={, ,}}%
\begin{sphinxVerbatim}[commandchars=\\\{\}]
maFonction      \PYG{c+c1}{\PYGZsh{} tout simplement ;)}
\end{sphinxVerbatim}


\chapter{Indices and tables}
\label{\detokenize{index:indices-and-tables}}\begin{itemize}
\item {} 
\DUrole{xref,std,std-ref}{genindex}

\item {} 
\DUrole{xref,std,std-ref}{modindex}

\item {} 
\DUrole{xref,std,std-ref}{search}

\end{itemize}



\renewcommand{\indexname}{Index}
\printindex
\end{document}